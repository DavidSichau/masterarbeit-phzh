\section{Datengrundlage}

Mit dieser Arbeit wurde ein erster Versuch durchgeführt zu zeigen, dass bestimmte Kompetenzen kontextunabhängig sind. Für eine Generalisierbarkeit der Resultate sind jedoch grössere Untersuchungsgruppen notwendig. Daher sollte ein erster Schritt dahin gehen, die Datengrundlage dieser Arbeit zu vergrössern. Damit sollten die bisher vorliegenden Hinweise stärker hervortreten und die Korrelationen besser abschätzbar sein.
Hierbei denke ich jedoch, dass die bisherigen Ergebnisse bestätigt werden und keine gegensätzlichen Resultate gefunden werden. Die bisherigen Ergebnisse sind jedoch aufgrund der zu geringen Stichprobe nicht generalisierbar. 




\section{Videoanalyse}

In dieser Arbeit wurde versucht zusätzliche Informationen mit Videoanalyse zu generieren. Dies ist nur sehr beschränkt gelungen, insbesondere da der Hauptfokus auf quantitativen Forschungsmethoden gelegt wurde und nicht auf qualitative. Dennoch hat sich gezeigt, dass die Ergebnisse der Videoanalyse nicht mit den über  Pen- und Paper-Tests erhobenen Daten übereinstimmen. Für genauere Analyse dieser Ergebnisse sollte der Fokus gezielt auf den Vergleich zwischen Videoanalyse und Pen- und Paper-Tests gelenkt werden. 

Interessant könnten die Videoanalysen insbesondere bei sprachlich schwächeren Schülern und Schülerinnen sein, welche aufgrund sprachlicher Schwierigkeiten in Pen- und Paper-Tests nur schwache Leistungen zeigen. Insbesondere hier könnten Videoanalysen helfen, festzustellen, ob es wirklich ein sprachliches Problem ist oder ob diese Schüler und Schülerinnen in diesen Tests auch tatsächlich schlechtere Leistungen erbringen.

\section{Methoden}

Auch methodisch wirft diese Arbeit weitere Fragen auf: Wie gezeigt ist insbesondere die Parameterschätzung des Rasch-Modells bei kleinen Datensätzen problematisch. Da die bisherigen Ansätze meistens Annahmen treffen, welche von kleinen Datensätzen nicht erfüllt werden können. Daher bräuchte es dringend neue Ansätze für die Schätzung der Personenparameter. Eine Möglichkeit wäre der marginale Maximum-Likelihood Schätzer. Dieser erfordert eine Annahme über die Verteilung der zugrundeliegenden Personenparametern. In den meisten existierenden Softwarepackages wird eine Normalverteilung angenommen \citep{Rost2004, Rizopoulos2006}. Diese Annahme einer Normalverteilung ist nicht festgelegt für den marginale Maximum-Likelihood Schätzer. Dieses Problem könnte vielleicht mit einem neuen Bootstrapping-Algorithmus, welcher mit den vorliegenden Daten eine Abschätzung über die Verteilung der Personenparametern macht, gelöst werden. Diese Abschätzung könnte dann als Initialisierung für den marginal Maximum-Likelihood Schätzer verwendet werden. Durch weitere Iterationen könnte dann das Ergebnis eventuell noch verbessert werden. Dies würde insbesondere bei kleinen Datensätzen das Rasch-Modell verbessern und nützlicher machen.

Ein weiteres grosses Problem der sozialwissenschaftlichen Forschung ist die geringe Auseinandersetzung mit den verwendeten Methodiken. Auch das Verwenden von closed-source Programmen ist sehr fragwürdig, da oft die Dokumentation nicht ausreichend ist, um die Ergebnisse nachvollziehen zu können (z.B. SPSS).  Meiner Meinung nach hat hier die Literatur in der sozialwissenschaftlichen Forschung grossen Nachholbedarf. So sollten die verwendeten Source-Codes für Analysen frei verfügbar gemacht werden (insbesondere bei Publikationen), damit andere Personen die Resultate nachvollziehen können, wie dies z.B. in der Bio-Informatik Standard ist. Es wurde versucht diese Forderung in dieser Arbeit umzusetzen, daher hier nochmals der Link zum vollständigen Source-Code der Auswertung.
\github{http://git.io/buGR}

