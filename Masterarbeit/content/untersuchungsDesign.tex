
\section{Anforderungen}

Um die vorliegende Fragestellung zu beantworten, ist es notwendig Test zu verwenden, welche die strategische Kompetenz des Messens messen. Zusätzlich müssen die Tests die Kompetenz des Messens unter verschiedenen Kontexten messen. Es wurden zwei existierende Test aus dem ExKonNawi Projekt verwendet. Der eine war aus dem Fachbereich Chemie bei dem eine Temperatur gemessen werden musste. Der zweite Test war aus dem Fachbereich Physik bei dem eine Kraft gemessen wurde. Zusätzlich wurde ein dritter Test entwickelt, bei welchem eine Temperatur Messung im Fach Physik durchgeführt wurde. Der dritte Test wurde so geplant, damit einmal der inhaltliche Kontext verändert werden kann (Kraftmessung vs Temperaturmessung), bei gleichem fachlichen Kontext und zum anderen der fachliche Kontext verändert werden kann ohne den inhaltlichen Kontext zu verändern. 


\section{Umsetzung}

Die Tests wurden zusammen mit einem Fragebogen an vier Klassen der Sek 1 A durchgeführt. In jeder Klasse wurden vier Gruppen gebildet, welche die Tests in unterschiedlicher Reihenfolge durchführten. Dafür gab es zwei Gründe. Zum einen war nur Material für 11 Tests verfügbar. Daher konnten die Tests nicht in voller Klassenstärke durchgeführt werden. Dies führte zur Bildung von zwei Gruppen, bei welcher eine zuerst den Fragebogen ausführte und die andere Gruppe den Fragebogen am Ende durchführte. Zusätzlich wurde noch der zweite und dritte Test in jeder Gruppe vertauscht um zu untersuchen ob eine Müdigkeit die Test-Ergebnisse verändern kann. Die Tabelle \ref{tab:Gruppenaufteilung} gibt eine Übersicht über die Gruppeneinteilung der Schülerinnen und Schüler innerhalb einer Klasse an.
\begin{table}[htbp]
  \centering
  \begin{tabular}{|p{3.1cm}|p{3.1cm}|p{3.1cm}|p{3.1cm}|}
  \hline Gruppe FABC & Gruppe FACB & Gruppe ABCF & Gruppe ACBF \\ 
  \hline Fragebogen (F) & Fragebogen (F) & Temperatur Physik (A) & Temperatur Physik (A) \\ 
  \hline Temperatur Physik (A) & Temperatur Physik (A) & Kraft Physik (B) & Temperatur Chemie (C) \\ 
  \hline Kraft Physik (B) & Temperatur Chemie (C) & Temperatur Chemie (C) & Kraft Physik (B) \\ 
  \hline Temperatur Chemie (C) & Kraft Physik (B) & Fragebogen (F) & Fragebogen (F) \\ 
  \hline 
  \end{tabular} 
  \caption{Aufteilung der Gruppen, innerhalb einer Klasse}
  \label{tab:Gruppenaufteilung}
\end{table}

Die vier Klassen waren alle von der selben Schulstufe jedoch in verschiedenen Gemeinden.


\begin{table}[htbp]
  \centering
  \begin{tabular}{|p{2.3cm}|p{3cm}|p{3cm}|p{3cm}|p{3cm}|}
  \hline & Klasse 1 & Klasse 2 & Klasse 3 & Klasse 4 \\ 
  \hline Ort & Glattbrugg & Glattbrugg & Stadt Zürich & Stadt Schaffhausen \\
  \hline Anzahl SuS & & & & \\
  \hline Datum der Durchführung & & & & \\
  \hline Uhrzeit & & & & \\
  \hline xyz & & & & \\
  \hline
  \end{tabular} 
  \caption{Aufteilung der Gruppen, innerhalb einer Klasse}
  \label{tab:Klassen}
\end{table}





\section{Durchführung}