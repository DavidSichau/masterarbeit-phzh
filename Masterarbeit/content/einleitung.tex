

In den letzten Jahrzehnten fand ein Wandel in den Bildungssystemen zu einer Output-Orientierung statt. Der Wandel wurde durch das schlechte Abschneiden einiger Länder in internationalen Studien wie PISA \citep{PISA-KonsortiumDeuschland2004} initialisiert. Dadurch fand ein Perspektivwechsel statt, sodass die Resultate des Bildungssystems überprüft werden und die Bildungssysteme nicht mehr über den Input gesteuert werden.

Dieser Wandel spiegelt sich auch in den neu entwickelten Bildungsstandards, in welchen die Kompetenzen der Schülerinnen und Schüler im Vordergrund stehen, welche diese nach dem Besuch des Bildungssystems erreicht haben sollen \citep{Oelkers2008}.

Oft werden diese angestrebten Kompetenzen unabhängig von einem inhaltlichen oder fachlichen Kontext gestellt (so z.B. bei HarmoS \citet{KonsotriumHarmoSNaturwissenschaften+2010}). Deshalb wird erwartet, dass die Kompetenzen generalisierbar sind und teilweise auf andere Situationen übertragen werden können \citep{Hartig2006}.

Diese Generalisierbarkeit und Transferbarkeit von Kontexten soll in dieser Masterarbeit genauer analysiert werden. Im Rahmen der Validierung von hands-on Testaufgaben im Projekt ExKoNawi der PH Zürich \citep{Metzger2013}, soll untersucht werden, inwiefern die Qualitätsstandards der Kompetenz des skalenbasierenden Messens vom Kontext abhängig sind. 

Dabei soll herausgefunden werden, ob das Erreichen eines bestimmten Qualitätsstandards von fachlichen oder inhaltlichen Kontext abhängig ist. Diese Fragestellung ist hinsichtlich der Kompetenzorientierung des Bildungssystems elementar.  Nicht transferierbare Kompetenzen wären von geringem Nutzen, da diese erworbenen Kompetenzen dann nur an einen genau definierten Kontext angewendet werden können. Dies ist jedoch nicht das Ziel des Bildungssystems, da die erworbene Kompetenzen auf ausserschulische Kontexte angewendet werden sollten.

Die Forschungsfrage, ob eine gewisse Kompetenz  von Lernenden auf der Sekundarstufe I in unterschiedlichen Kontexten gleich verfügbar ist, soll mithilfe von mehreren hands-on Experimentiertests zu einer Kompetenz in unterschiedlichen Kontexten beantwortet werden. Bevor diese Frage jedoch beantwortet werden kann, soll ein Überblick über den theoretischen Hintergrund zum Kontext, zum einen basierend auf dem Begriff des Transfers und zum anderen basierend auf dem Kompetenzbegriff, gegeben werden.









