

Nachdem im letzten Kapitel die Ergebnisse präsentiert wurden, soll in diesem Kapitel versucht werden mit Hilfe der Ergebnisse die Forschungsfrage zu beantworten.

\section{Kodierung}

\subsection{Items}

Da sowohl die Qualitätsstandards als auch die Niveaus auf den Items basieren, ist eine gute Kodierung derselbigen elementar für diese. Durch die Zweitkodierung der Items sollte sichergestellt werden, dass die Kodierung der Items verlässlich und wiederholbar ist. In Tabelle \ref{tab:CohenKappa} sind die Ergebnisse für die Interrater-Reliabilität aufgeführt. Bis auf wenige Ausnahmen befinden sich alle Werte oberhalb von $\kappa > 0.75$ was nach \citet[S.111]{Greve1997} sehr gut bis ausgezeichnet ist. \citet{Landis1977} bezeichnet jedoch auch die niedrigen $\kappa$-Werte bei denen $\kappa > 0.61$ ist als "`substantial strength of agreement'. 

Ein Problem bei der Kodierung der Items und der Überprüfung, war jedoch, dass viele Schülerinnen und Schüler bestimmte Items nicht erreichten. Daher konnte Cohen's $\kappa$ nicht für alle Items berechnet werden. Da die prozedurale Übereinstimmung dort jedoch sehr hoch war, kann auch bei diesen Items von einer korrekten Kodierung ausgegangen werden. Dieses Problem kann auch eine Erklärung für die sehr gute Übereinstimmung bei bestimmten Items sein. So war es meistens sehr klar, wenn ein Schüler oder eine Schülerin ein Item nicht erreicht hatten. Daher war die Kodierung meistens sehr eindeutig.

Aufgrund dieser Ergebnisse kann davon ausgegangen werden, dass die Zweitkodierung aller Schülerinnen und Schüler keine deutlich abweichende Resultate geliefert hätten und daher die Zweitkodierung von 15\% der Schülerinnen und Schüler ausreichend war um die Qualität und Reliabilität der Kodierung festzustellen.

Daher kann davon ausgegangen werden, dass die Reliabilität der Kodierung gegeben ist und die Kodierung korrekt und nachvollziehbar ist.

\subsection{Qualitätsstandards}

Ein Problem bei der Definition der Qualitätsstandards ist die Unterschiedliche Definition in der Literatur. So verwendete \citet{Gut2013a} noch eine andere Reihenfolge der Qualitätsstandards. Die in dieser Arbeit verwendete Reihenfolge der Qualitätsstandards basiert auf den Arbeiten von \citet{Metzger2013, Hild2014a}. Ein Problem dabei ist jedoch, dass die Schwellenwerte für das Erreichen der Qualitätsstandards nicht publiziert sind. Die Schwellenwerte wurde daher von internen Dokumenten von Pitt Hild übernommen.

Die erreichten Qualitätsstandards in Tabelle \ref{tab:QS} zeigen, dass insbesondere die Qualitäts-standards 3, 4 und 5 nur von einem geringen Prozentsatz der Schülerinnen und Schüler erreicht werden. Und es auch einen Unterschied in den erreichten Qualitätsstandards zwischen den einzelnen Test gibt. In dieser Arbeit wird nicht auf diese Unterschiede eingegangen. Dafür sei auf folgende Arbeit hingewiesen \citet{Sichau2015}. Diesen Unterschied in den erreichten Qualitätsstandards deckt sich jedoch mit den Ergebnissen von \citet{Metzger2013}.

\subsection{Niveaus}

Dieses schlechte Abschneiden der Klassen spiegelt sich auch in den erreichten Niveaus wieder. So sieht man in Tabelle \ref{tab:Niveau}, dass ein Grossteil der Schülerinnen und Schüler nicht über das Niveau 2 hinauskommen, sowohl beim unbedingten als auch beim bedingten Niveau. Im Vergleich zu \citet{Metzger2013} scheiden die Schülerinnen und Schüler in der 7. Klasse schlechter ab. 

Da leider der Zeitpunkt der Datenerhebung in der Arbeit von \citet{Metzger2013} nicht aufgeführt ist, ist nicht klar ob der frühe Zeitpunkt des Testes (beginn des ersten Halbjahres) einen eventuellen Einfluss auf das Abschneiden der Schülerinnen und Schüler hatte. So war dies bei allen Klassen bei denen diese Tests durchgeführt wurden, das erste Mal, dass sie in der Oberstufe experimentiert haben. Auch kannten die Schülerinnen und Schüler den Kraftmesser nicht und konnten nur durch ausprobieren herausfinden, wie dieser funktioniert. Daher ist die Vermutung, dass wenn der Test im zweiten Halbjahr der 7. Klasse durchgeführt wurde ein deutlich besseres Resultat erzielt werden könnte.



\section{Fragebogen}

Die verwendeten Fragen im Fragebogen aus SESSKO \citep{Schone2002} und die abgewandelten Fragen nach \citet{Dierks2014} wurden aus innere Konsistenz überprüft. Beide Skalen erreichten wie in \ref{txt:Cronbach} beschrieben eine sehr gute innere Konsistenz, insbesondere da Cronbach's $\alpha$ eher zu einer Unterschätzung der inneren Konsistenz führt \citep{Eisinga2013}. Auch durch das Weglassen einzelner Fragen würde die innere Konsistenz nicht verbessert werden (siehe Tabelle \ref{tab:SESSKO} und Tabelle \ref{tab:NatSK}). Daher kann angenommen werden, dass beide Skalen das jeweilige Selbstkonzept konsistent widerspiegeln und ausreichend Fragen zu jeder Skala vorhanden sind. 

Der Mittelwert aller Schülerinnen und Schüler beim "`Schulisches Selbstkonzept - absolut"' kann mit den Werten aus der Literatur \citep{Schone2002} verglichen werden. Dabei hat die hier untersuchte Schülergruppe ein leicht überdurchschnittliches Selbstkonzept verglichen mit der Referenzgruppe (4. - 10. Klasse in verschiedenen Deutsch Schulformen und Bundesländern.). Der Grund dafür könnte der erst kürzlich erfolgte Übertritt auf die Oberstufe und dort die Einteilung in die Sek A sein. 

\section{Unterschied zwischen den Klassen}

Vor der weiteren Analyse der Daten muss erst festgestellt werden, ob die Datensätze der einzelnen Klassen kombiniert werden dürfen. Wichtig ist dabei, dass der exakte Test nach Fischer verwendet wird und nicht der Chi-Quadrat-Test, da bei kleinen Datensätzen (wie dem hier Vorliegenden) der Chi-Quadrat-Test nicht geeignet ist \citep{Mehta1984}.

Für den exakten Fischer-Test wurden die erreichten Qualitätsstandards in den einzelnen Klassen verglichen. Die Qualitätsstandards wurden verwendet, da im Vergleich zu den Items das statistische Rauschen geringer ist und gleichzeitig nicht viel an Information verloren geht. Aus der Tabelle \ref{tab:KlassenVergleiche}, kann geschlossen werden, dass kein signifikanter Unterschied zwischen den einzelnen Klassen existiert, da alle p-Werte über 0.05 liegen.

Es dürfen daher alle Datensätze kombiniert werden, da das Erreichen eines Qualitäts-standards nicht davon abhängt in welcher Klasse ein Schüler oder eine Schülerin ist. Für alle weiteren Analysen wurden daher alle Datensätze kombiniert und nicht nach Klassen unterschieden.

\section{Ist das Abschneiden in den Tests unterschiedlich}

Nachdem gezeigt wurde, dass der ganze Datensatz insgesamt analysiert werden kann, wurde versucht die Forschungsfrage zu beantworten. Dafür ist es notwendig festzustellen, ob das Erreichen der Qualitätsstufen zwischen den unterschiedlichen Tests signifikant unterschiedlich ist.