\chapter{Theorie und Problemlage}

\section{Transfer}

Es wird erwartet, dass in der Schule vermitteltes Wissen universell aufgerufen werden kann und auch das Gelernte auf andere Kontexte angewendet werden kann. Dieses universell verfügbare Wissen ist eng mit dem Begriff des Transfers verknüpft. \citet{Greeno1996} definierten Transfer als "`the process of applying knowlege in new situations"'. Aber auch innerhalb der schulischen Bildung gibt es einen Transfer zwischen den verschiedenen Fächern. So wird von SuS erwartet, dass Fähigkeiten, Lösungsstrategien, Konzepte und anderes Wissen auf andere Fächer übertragen werden soll und dort abgerufen werden kann.

\subsection{Historischer Überblick}


\subsubsection{Woodworth 1901}

Eines der ersten Experimente zu Transfer wurde von \citet{Woodworth1901} gemacht. Dabei mussten Probanden die Grösse von Rechtecken schätzen. Nachdem die Personen durch Wiederholungen sich verbessert hatten wurde ihnen zwei neue Test Sets gegeben. In einem gab es neue Rechtecke, welche im ursprünglichen Set nicht enthalten waren. Die zweite Gruppe bekam Sets bei denen andere Formen enthalten waren (z. B. Kreise und Dreiecke). Die zweite Testgruppe machte ähnlich viel Fehler, wie vor dem Training mit den Rechtecken. Daraus schloss \citeauthor{Woodworth1901}, dass kein Transfer stattgefunden haben kann.
Ein Beispiel für eine Untersuchung auf universitärem Niveau ist \citet{Renkl1994}. Er konnte zeigen, dass Nichtökonomen eine simulierte Firma besser führten, als Studierende der Betriebswissenschaften kurz vor Ihrem Abschluss. Diese Resultate führen zu dem Schluss, dass Transfer nur sehr schwierig erreicht werden kann, und wenn oft nur unter sehr ähnlichen Bedingungen.
Dieses Transfers Verständnis basiert und stützt das Reiz-Reaktions-Modell des Lernens \citep{Detterman1993, Mietzel2007}.
\subsubsection{Ferguson 1956}
Eine alternative Theorie zum Transfer wurde von \citet{Ferguson1956} entwickelt. Fergusons Theorie basiert darauf, dass die Intelligenz einer Person sich auf deren Transferleistung auswirkt. So findet nach \citet{Ferguson1956} bei dem Lernen permanent ein Transfer statt, da jede Lernaufgabe von der anderen unterschiedlich ist und daher Transfer stattfinden muss. Im Unterschied zu \citet{Woodworth1901} betrachtet \citeauthor{Ferguson1956} Transfer als einen kontinuierlichen Prozess, welcher durch Lernen verbessert werden kann. Wichtig ist jedoch zu beachten, dass \citeauthor{Ferguson1956} Theorie nur Nah-Transfer beschreibt. Unter Nah-Transfer wird Transfer zwischen sehr ähnlichen Situationen definiert. 
\subsubsection{Judd 1908}
Eine der grundlegenden Studien zu Fern-Transfer, bei welchem erworbenes Wissen auf Kontexte angewendet werde soll, welche sich deutlich vom Kontext, unter welchem das Wissen erworben wurde, unterschieden, wurde von \citet{judd1908} gemacht. Im Vergleich zu \citeauthor{Woodworth1901} geht Judd davon aus das der Unterschied zwischen den beiden Situationen nicht nur abhängig von der Ähnlichkeit un den Unterschieden zwischen den beiden Situationen ist, sondern auch davon abhängt wie die erste Situation gelernt wurde. Um dies zu belegen führte Judd eine sehr bekannte Studie durch. Bei dieser wurden Kinder genommen, welche mit einem Dart auf eine Zielscheibe unter Wasser werfen sollten. Beide Gruppen bekamen zu beginn die Möglichkeit dies zu trainieren. Später wurde das Werfen wiederholt, wobei die Postition der Zielscheibe jedoch unterschiedlich war, und untersucht welche Gruppe besser war. Eine der beiden trainierten Gruppen wurde währen sie die Situation A trainierten erklärt, warum die Scheibe so schwierig zu treffen war. Ihnen wurde also das Prinzip der Lichtbrechung erklärt. Die Gruppe welche die Erklärung bekommen hatte schnitt unter der neuen Situation deutlich besser ab, als die andere Gruppe. \citet{judd1908} erklärte dieses damit, dass die einen wussten, welches Prinzip sie auch bei der zweiten Situation anwenden können. Denen das Prinzip nicht erklärt wurden, haben gelernt ihren Wurf auf die erste Situation anzuwenden, konnten dieses Wissen jedoch nicht generalisieren, da dies spezifisch für die Situation erworben wurde. Im Vergleich zu \citeauthor{Woodworth1901} beinhaltet die Theorie von \citeauthor{judd1908} einen kognitivistisches Verständnis des Lernens. Da die Lernenden ein immer besseres Verständnis der Welt um sich selbst konstruieren und so neue Situation basierend auf ihrer internen Repräsentation der Welt lösen können. \citet{Detterman1993} kritisiert an dieser Studie die Verwendung von Transfer. So erklärt \citeauthor{judd1908} einem Teil der Personen das zugrunde liegende Prinzip. Dies ist laut \citeauthor{Detterman1993} jedoch so, wie wenn man den Personen sagen würde, dass sie dieses Prinzip verwenden sollen. Was dann identisch wäre, wie wenn man einer Anleitung folgen würde.

\subsection{Gick und Holyoak 1980}

Eine weitere bedeutende Studie zu Transfer wurde von \citet{Gick1980} durchgeführt. Dabei wurde untersucht, unter welchen Bedingungen Lernende Analogien verwenden um strukturell ähnliche Probleme zu lösen. Ein Beispiel Problem welches sie den Lernenden gaben war, wie kann ein Tumor mit Strahlung zerstört werden ohne dass gesundes Gewebe geschädigt wird. Dieses Problem wurde erstmals von \citet{Duncker1945} verwendet. Dieses Problem kann gelöst werden, indem man mehrere Strahlen verwendet, welche sie nur im Tumor überlagern. Bevor sie dieses Problem lösten bekommen Sie eine Geschichte erzählt, bei welcher das gleiche Prinzip verwendet wird. In dieser Geschichte ging es darum ein Fort das von Minen umgeben ist zu erobern. Durch aufteilen der Angreifer in mehrere angreifende Gruppen, die unterschiedliche Wege gehen, wurde die Belastung auf die Minen reduziert und das Fort konnte erobert werden. Das Resultat dieser Studie zeigte, dass spontaner Transfer nur sehr selten stattfindet. Das Hören der Geschichte führt nicht zu einer höheren Wahrscheinlichkeit das zweite ähnliche Problem zu lösen, solange die Lernenden nicht auf die Ähnlichkeit aufmerksam gemacht werden.

\subsection{Kritiken}

\citet{Lave1988} kritisiert die verschieden hier vorgestellten Untersuchungen. Da bei allen angenommen wird, das  Wissen automatisch generalisierbares Wissen erzeugt, welches auf verschiedene Situationen angewendet werden kann. Sie schlägt eine Alternative vor welche sie als "'practice view"' bezeichnet. Bei dieser wird Wissen von Personen erworben, welche an speziellen Übungen teilnehmen und daraus nur Wissen entwickelt wird, welches auf diese spezifische Situation zutrifft.

Folgende Kritiken erhebt sie. So stellt sie die Frage was lernen die Teilnehmer der verschiedenen Studien überhaupt. So greift sie insbesondere die Annahme an, dass die Teilnehmer der Studien Kontext unabhängig lernen. Sie lernen immer Kontext spezifisch. Ein anderer Punkt welche Sie angreift ist, wer definiert die Ähnlichkeit der Probleme. Ist die Ähnlichkeit der Probleme für die Teilnehmer der Studie auch greifbar. Auch \citet{Detterman1993} kritisiert die Studien. So sollten seiner Meinung alle Studien zu Transfer als Doppel-Blind Studien durchgeführt werden, da der Studienleiter unbewusst die Leistung der Probanden ändern kann. \citet{Detterman1993} fordert daher:
\begin{quote}
No tranfer experiment should be carried out without using a double blind procedure, particularly experiments assessing general transfer \citet[S. 10]{Detterman1993}.
\end{quote}

\subsection{Definition von Transfer}

Nachdem einige grundlegenden Studien zu Transfer exemplarisch aufgezeigt wurden, soll nun der Begriff des Transfers genauer definiert werden. 


\citet{Lobato2002a} möchte einen Kritik-Punkt von \citet{Lave1988} lösen. So kritisierte \citeauthor{Lave1988}, das der Untersucher festlegt was Transfer ist. So legten sie als Messung für den Transfer fest, welche Ähnlichkeit der Proband selbst zwischen verschiedenen Situationen zieht. So untersuchten Sie einen Schüler, welcher eine Rollstuhlrampe erhöhen sollte ohne die Steigung zu verändern. Der Schüler löste dieses Problem in dem er die Verhältnisse von Höhe zu Länge konstant hielt. Er verwendete nicht die im Mathematik Unterricht gelernten Formeln für dieses Problem. In den bisherigen Untersuchungen wäre daher angenommen, dass der Schüler keinen Transfer geleistet hat. Aufgrund der Interviews stellte sie jedoch fest, dass der Schüler sehr wohl Transfer geleistet hat, indem er das Konzept von konstanter Geschwindigkeit als Verhältnis von zurückgelegter Strecke zur Zeit auf dieses Problem angewendet hat.

\citet{Lobato2002a} konnten damit zeigen, dass wenn man die Ähnlichkeit zwischen zwei verschiedenen Situationen nicht mit strukturellen Ähnlichkeiten oder Unterschieden beschrieben werden sollen. Sondern damit, wie der Lernende die Ähnlichkeiten zwischen den Situationen wahr nimmt.

\subsection{Elemente von Transfer}

Nachdem ein Überblick über die historische Entwicklung von Transfer gegeben wurde und auch die aktuelle Definition diskutiert wurde soll nur auf die grundlegenden Elemente welche bei Transfer anzutreffen sind eingegangen werden.

\citet{Marini1995} definiert drei Elemente, welche zu einem Transfer führen. Das erste Element sind Merkmale des Lernenden. Dieser hat sobald er eine Situation antrifft bereits ein bestimmtes prozedurales und deklaratives Wissen, welches er sich erarbeitet hat und abrufen kann. In einem bestimmten Kontext kann er einen teil davon abrufen und anwenden \citep[s. S. 189ff]{Marini1995}. Dies führt dazu das lösungsrelevantes Wissen von den Vorhanden und dem verarbeitbarem Wissen abhängt. Zusätzlich kann in einem bestimmten Kontext jedoch nicht alles Wissen abrufen werden, da man mit trägem Wissen rechnen muss und auch der aktuellen Motivation des Lernenden. 

Als zweites Element von Transfer gibt \citeauthor{Marini1995} die Merkmale einer Aufgabenstellung an. So hängt Transfer von der Ähnlichkeit der Aufgabe ab. Dabei gibt es jedoch einen Unterschied zwischen Novizen und Experten. Novizen vergleichen Aufgaben hauptsächlich aufgrund oberflächlicher Merkmale, wohingegen Experten sich auf die zugrunde liegenden Prinzipien fokussieren \citep[s. S. 279]{Marini1995}. Aufgrund dessen, haben Novizen oft Problem den Zusammenhang zwischen Aufgaben zu sehen und können daher keinen Transfer durchführen.

Das dritte Element ist der Kontext in den ein Problem eingebettet ist. Ein Beispiel dafür ist die Untersuchung von \citet{Godden1975}. Dort lernten Taucher Wörter Unterwasser auswendig. Bei einer späteren Überprüfung konnten sie sich an mehr Wörter erinnern, wenn es Unterwasser wiederholt wurde im Vergleich zu einer Wiederholung auf dem Festland. Dieser Ortswechsel ist auch bei ausserschulischem Kontext gegeben. Aber auch innerhalb der Schule kann es zu unterschieden kommen. Ein Beispiel dafür liefert \citet{Schoenfeld1988} so hatten Lernende keine Schwierigkeiten mit einer Divisionsaufgabe. Wenn die Aufgabe jedoch in einen Kontext gestellt wurde, in eine Textaufgabe eingebettet, scheiterten die meisten der Lernenden. 

Erst durch die Berücksichtigung aller drei Elemente lässt sich Transfer ganzheitlich Betrachten. Nicht wie  \citet{Woodworth1901}, welcher nur den Aspekt der Aufgabenmerkmale genauer untersucht hat. Erst neuere Arbeiten berücksichtigen alle Elemente und insbesondere den Kontext \citep{Lobato2002a, Detterman1993, Greeno1996}

\subsection{Konsequenzen für den Unterricht}

Wie \citet{claxton1990} zeigte, darf jedoch nicht davon ausgegangen werden, dass in der Schule erworbenes Wissen ohne weiteres auf andere Alltags Probleme angewendet werden können. So sprach \citet{Whitehead1929} von "`trägem Wissen"' (inert knowledge), wenn Wissen vorhanden ist um ein Problem zu lösen, dieses jedoch nicht automatisch abgerufen werden kann. Dieses Wissen ist erst greifbar, wenn die Person angeregt wird dieses Wissen zu verwenden. Nach \citet{Whitehead1929} entsteht träges Wissen oft unter schulischen oder universitären Bedingungen. 

Wie muss nun schulischer Unterricht aussehen, welcher verhindert, das träges Wissen entsteht und möglichst viel Transfer von Wissen stattfinden kann. \citet[s. S. 316ff]{Mietzel2007} schlägt verschiedene Strategien vor wie dies verhindert werden kann.

\section{Kompetenz}

Nachdem ein Überblick über den Transfer erarbeitet wurde, soll in diesem Abschnitt versucht werden die Konsequenzen aus Transfer mit dem Kompetenzbegriff zu verknüpfen.

