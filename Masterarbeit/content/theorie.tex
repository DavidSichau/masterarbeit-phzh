\chapter{Theorie und Problemlage}

Für die Beantwortung und Entwicklung der Forschungsfrage ist es wichtig den Begriff des Kontextes zu definieren. Zu Beginn wird der Begriff des Kontextes aus dem Sichtwinkel des Transfers untersucht. In einem zweiten Teil wird der Kontext basierend auf dem Kompetenzbegriff analysiert. Zuletzt werden die Erkenntnisse gesammelt und die genaue Forschungsfrage dieser Masterarbeit definiert.

\section{Transfer}

Es wird erwartet, dass in der Schule vermitteltes Wissen universell aufgerufen werden kann und dass das Gelernte auf andere Kontexte angewendet werden kann. Dieses universell verfügbare Wissen ist eng mit dem Begriff des Transfers verknüpft. \citet{Greeno1996} definierten Transfer als "`the process of applying knowlege in new situations"'. Aber auch innerhalb der schulischen Bildung gibt es einen Transfer zwischen den verschiedenen Fächern. So wird von Schülerinnen und Schüler erwartet, dass Fähigkeiten, Lösungsstrategien, Konzepte und anderes Wissen auf andere Fächer übertragen werden soll und dort abgerufen werden kann.

\subsection{Historischer Überblick}

Um den Begriff des Kontextes im Zusammenhang mit Transfer zuverorten, soll zuerst ein historischer Überblick über den Begriff des Transfers gegeben werden.


\subsubsection{Woodworth 1901}

Eines der ersten Experimente zu Transfer wurde von \citet{Woodworth1901} gemacht. Dabei mussten Probanden die Grösse von Rechtecken schätzen. Nachdem die Personen sich durch Wiederholungen verbessert hatten, wurde ihnen zwei neue Test Sets gegeben. In einem gab es neue Rechtecke, welche im ursprünglichen Set nicht enthalten waren. Die zweite Gruppe bekam Sets bei denen andere Formen enthalten waren (z. B. Kreise und Dreiecke). Die zweite Testgruppe machte ähnlich viel Fehler, wie vor dem Training mit den Rechtecken. Daraus schloss \citeauthor{Woodworth1901}, dass kein Transfer stattgefunden haben kann.

Ein ähnliches Resultat auf universitärem Niveau konnte von \citet{Renkl1994} gezeigt werden. Er konnte zeigen, dass Nichtökonomen eine simulierte Firma besser führten, als Studierende der Betriebswissenschaften kurz vor ihrem Abschluss. Diese Resultate führen zu dem Schluss, dass Transfer nur sehr schwierig erreicht werden kann, und wenn oft nur unter sehr ähnlichen Bedingungen. Die \citeauthor{Woodworth1901} Theorie zu Transfer basiert auf der Idee von identischen Elementen \citep{Pea2013b}. In diesem Theorie-Verständnis entsteht Transfer, wenn Wissen auf zwei verschiedenen Aufgaben, welche jedoch identische Merkmale/Elemente besitzen, angewendet wird. Dieses Transfers Verständnis basiert und stützt das Reiz-Reaktions-Modell des Lernens \citep{Detterman1993, Mietzel2007}.


\subsubsection{Ferguson 1956}
Eine alternative Theorie zum Transfer wurde von \citet{Ferguson1956} entwickelt. Fergusons Theorie basiert darauf, dass die Intelligenz einer Person sich auf deren Transferleistung auswirkt. So findet nach \citet{Ferguson1956} bei dem Lernen permanent ein Transfer statt, da jede Lernaufgabe von der anderen unterschiedlich ist und daher Transfer stattfinden muss. Im Unterschied zu \citet{Woodworth1901} betrachtet \citeauthor{Ferguson1956} Transfer als einen kontinuierlichen Prozess, welcher durch Lernen verbessert werden kann. Wichtig ist jedoch zu beachten, dass \citeauthor{Ferguson1956} Theorie nur Nah-Transfer beschreibt. Unter Nah-Transfer wird Transfer zwischen sehr ähnlichen Situationen definiert. 


\subsubsection{Judd 1908}
Eine der grundlegenden Studien zu Fern-Transfer, bei welchem erworbenes Wissen auf Kontexte angewendet werde soll welche sich deutlich vom Kontext, unter welchem das Wissen erworben wurde, unterschieden, wurde von \citet{judd1908} gemacht. Im Vergleich zu \citeauthor{Woodworth1901} geht Judd davon aus, dass der Unterschied zwischen den beiden Situationen nicht nur abhängig von der Ähnlichkeit und den Unterschieden zwischen den beiden Situationen ist, sondern auch davon abhängt, wie die erste Situation gelernt wurde. Um dies zu belegen führte Judd eine sehr bekannte Studie durch. Bei dieser wurden Kinder genommen, welche mit einem Dart auf eine Zielscheibe unter Wasser werfen sollten. Beide Gruppen bekamen zu beginn die Möglichkeit dies zu trainieren. Später wurde das Werfen wiederholt, wobei die Position der Zielscheibe jedoch unterschiedlich war, und untersucht welche Gruppe besser war. Eine der beiden trainierten Gruppen wurde währen sie die Situation A trainierten erklärt, warum die Scheibe so schwierig zu treffen war. Indem ihnen zusätzlich zum Training noch das Prinzip der Lichtbrechung erklärt wurde. Die Gruppe welche die Erklärung bekommen hatte schnitt unter der neuen Situation deutlich besser ab, als die andere Gruppe. \citet{judd1908} erklärte dieses damit, dass die einen wussten, welches Prinzip sie auch bei der zweiten Situation anwenden können. Die Schülerinnen und Schüler welchen das Prinzip nicht erklärt wurden, haben gelernt ihren Wurf auf die erste Situation anzuwenden, konnten dieses Wissen jedoch nicht generalisieren, da dies spezifisch für die Situation erworben wurde. 

Im Vergleich zu \citeauthor{Woodworth1901} beinhaltet die Theorie von \citeauthor{judd1908} einen kognitivistisches Verständnis des Lernens. Da die Lernenden ein immer besseres Verständnis der Welt um sich selbst konstruieren und so neue Situation basierend auf ihrer internen Repräsentation der Welt lösen können. \citet{Detterman1993} kritisiert an dieser Studie die Verwendung von Transfer. So erklärt \citeauthor{judd1908} einem Teil der Personen das zugrunde liegende Prinzip. Dies ist laut \citeauthor{Detterman1993} jedoch äquivalent, wie wenn man den Personen sagen würde, dass sie dieses Prinzip verwenden sollen. Was dann identisch wäre, wie wenn man einer Anleitung folgen würde.

\subsubsection{Gick und Holyoak 1980}

Eine weitere bedeutende Studie zu Transfer wurde von \citet{Gick1980} durchgeführt. Dabei wurde untersucht, unter welchen Bedingungen Lernende Analogien verwenden, um strukturell ähnliche Probleme zu lösen. Ein Beispiel Problem welches sie den Lernenden gaben war, wie kann ein Tumor mit Strahlung zerstört werden, ohne dass gesundes Gewebe geschädigt wird. Dieses Problem wurde erstmals von \citet{Duncker1945} verwendet. Dieses Problem kann gelöst werden, indem man mehrere Strahlen verwendet, welche sie nur im Tumor überlagern. Bevor sie dieses Problem lösten bekommen die Lernenden eine Geschichte erzählt, bei welcher das gleiche Prinzip verwendet wird. In dieser Geschichte ging es darum ein Fort, welches von Minen umgeben ist zu erobern. Durch aufteilen der Angreifer in mehrere angreifende Gruppen, die unterschiedliche Wege gehen, wurde die Belastung auf die Minen reduziert und das Fort konnte erobert werden. Das Resultat dieser Studie zeigte, dass spontaner Transfer nur sehr selten stattfindet. Das Hören der Geschichte führt nicht zu einer höheren Wahrscheinlichkeit das zweite ähnliche Problem zu lösen, solange die Lernenden nicht auf die Ähnlichkeit aufmerksam gemacht werden.

\subsection{Kritiken}

\citet{Lave1988} kritisiert die verschieden hier vorgestellten Untersuchungen. Da bei allen angenommen wird, das  Wissen automatisch generalisierbares Wissen erzeugt, welches auf verschiedene Situationen angewendet werden kann. Sie schlägt eine Alternative vor welche sie als "'practice view"' bezeichnet. Bei dieser wird Wissen von Personen erworben, welche an speziellen Übungen teilnehmen und daraus nur Wissen entwickelt wird, welches auf diese spezifische Situation(Kontext) zutrifft.

Folgende Kritiken erhebt sie. So stellt sie die Frage was lernen die Teilnehmer der verschiedenen Studien überhaupt. So greift sie insbesondere die Annahme an, dass die Teilnehmer der Studien Kontext unabhängig lernen. Sie lernen immer Kontext spezifisch. Ein anderer Punkt welche Sie angreift ist, wer definiert die Ähnlichkeit der Probleme. Ist die Ähnlichkeit der Probleme für die Teilnehmer der Studie auch greifbar. Auch \citet{Detterman1993} kritisiert die Studien. So sollten seiner Meinung alle Studien zu Transfer als Doppel-Blind Studien durchgeführt werden, da der Studienleiter unbewusst die Leistung der Probanden ändern kann. \citeauthor{Detterman1993} fordert daher:
\begin{quote}
No tranfer experiment should be carried out without using a double blind procedure, particularly experiments assessing general transfer \citet[S. 10]{Detterman1993}.
\end{quote}





\subsection{Lobato und Sievert 2002}



Nachdem einige historische Studien zu Transfer exemplarisch aufgezeigt wurden, soll eine aktuelle Studie zu Transfer welche auf die Kritiken eingeht gezeigt werden.


\citet{Lobato2002a} möchte einen Kritik-Punkt von \citet{Lave1988} lösen. So kritisierte \citeauthor{Lave1988}, dass der Untersucher festlegt was Transfer von Wissen ist. Daher legten \citet{Lobato2002a} als Messung für Transfer fest, welche Ähnlichkeit der Proband selbst zwischen verschiedenen Situationen zieht. So untersuchten Sie einen Schüler, welcher eine Rollstuhlrampe erhöhen sollte ohne die Steigung zu verändern. Der Schüler löste dieses Problem in dem er die Verhältnisse von Höhe zu Länge konstant hielt. Er verwendete dafür jedoch nicht die im Mathematik Unterricht gelernten Formeln. In den bisherigen Untersuchungen wäre daher angenommen worden, dass der Schüler keinen Transfer geleistet hat. Aufgrund der Interviews stellte sie jedoch fest, dass der Schüler sehr wohl Transfer geleistet hat, indem er das Konzept von konstanter Geschwindigkeit als Verhältnis von zurückgelegter Strecke zur Zeit auf dieses Problem angewendet hatte.

\citet{Lobato2002a} konnten damit zeigen, dass wenn man die Ähnlichkeit zwischen zwei verschiedenen Situationen(Kontexten) nicht mit strukturellen Ähnlichkeiten oder Unterschieden beschrieben werden sollen. Sondern damit, wie der Lernende die Ähnlichkeiten zwischen den Situationen(Kontexten) wahr nimmt.

\subsection{Elemente von Transfer}

Nachdem ein Überblick über die historische Entwicklung von Transfer gegeben wurde, soll nun auf die grundlegenden Elemente, welche bei Transfer anzutreffen sind eingegangen werden.

\citet{Marini1995} definieren drei Elemente, welche zu einem Transfer führen. Das erste Element besteht aus Merkmale des Lernenden. Dieser hat sobald er eine Situation antrifft bereits ein bestimmtes prozedurales und deklaratives Wissen, welches er sich erarbeitet hat und abrufen kann. In einem bestimmten Kontext kann er einen Teil davon abrufen und anwenden \citep[s. S. 189ff]{Marini1995}. Dies führt dazu das lösungsrelevantes Wissen von den Vorhanden und dem verarbeitbarem Wissen abhängt. Zusätzlich kann in einem bestimmten Kontext jedoch nicht alles Wissen abgerufen werden, da man mit trägem Wissen rechnen muss und auch der aktuellen Motivation des Lernenden. 

Als zweites Element von Transfer gibt \citeauthor{Marini1995} die Merkmale einer Aufgabenstellung an. So hängt Transfer von der Ähnlichkeit der Aufgabe ab. Dabei gibt es jedoch einen Unterschied zwischen Novizen und Experten. Novizen vergleichen Aufgaben hauptsächlich aufgrund oberflächlicher Merkmale, wohingegen Experten sich auf die zugrunde liegenden Prinzipien fokussieren \citep[s. S. 279]{Marini1995}. Aufgrund dessen, haben Novizen oft Problem den Zusammenhang zwischen Aufgaben zu sehen und können daher keinen Transfer durchführen.

Das dritte Element ist der Kontext in den ein Problem eingebettet ist. Ein Beispiel dafür ist die Untersuchung von \citet{Godden1975}. Dort lernten Taucher Wörter Unterwasser auswendig. Bei einer späteren Überprüfung konnten sie sich an mehr Wörter erinnern, wenn es Unterwasser wiederholt wurde im Vergleich zu einer Wiederholung auf dem Festland. Dieser Ortswechsel ist auch bei ausserschulischem Kontext gegeben. Aber auch innerhalb der Schule kann es zu unterschieden kommen. Ein Beispiel dafür liefert \citet{Schoenfeld1988} so hatten Lernende keine Schwierigkeiten mit einer Divisionsaufgabe. Wenn die Aufgabe jedoch in einen Kontext gestellt wurde, wie z.B. in eine Textaufgabe eingebettet wurde, scheiterten die meisten der Lernenden. 

Erst durch die Berücksichtigung aller drei Elemente lässt sich Transfer ganzheitlich Betrachten. Nicht wie  \citet{Woodworth1901}, welcher nur den Aspekt der Aufgabenmerkmale genauer untersucht hat. Erst neuere Arbeiten berücksichtigen alle Elemente und insbesondere den Kontext \citep{Lobato2002a, Detterman1993, Greeno1996}

\subsection{Konsequenzen für den Unterricht}
\label{sec:TransferUnterricht}

Wie \citet{claxton1990} zeigte, darf jedoch nicht davon ausgegangen werden, dass in der Schule erworbenes Wissen ohne weiteres auf andere Alltags Probleme angewendet werden können. So sprach \citet{Whitehead1929} von "`trägem Wissen"' (inert knowledge), wenn Wissen vorhanden ist um ein Problem zu lösen, dieses jedoch nicht automatisch abgerufen werden kann. Dieses Wissen ist erst greifbar, wenn die Person angeregt wird dieses Wissen zu verwenden. Nach \citet{Whitehead1929} entsteht träges Wissen oft unter schulischen oder universitären Bedingungen. 


\citeauthor{Detterman1993} geht sogar noch weiter und schliesst aus den Studien zu Transfer:
\begin{quote}
that, if you want people to learn something, teach it to them. Don't teach them something else and expect them to figure out what you really want them to do \citep[S. 21]{Detterman1993}.
\end{quote}

Andere Autoren haben jedoch keine so pessimistische Sicht auf die Fähigkeit zu Transfer und geben Empfehlungen, wie schulischer Unterricht aussehen muss, welcher verhindert, dass träges Wissen entsteht und möglichst viel Transfer von Wissen stattfinden kann.

\subsubsection{Überlernen von Fähigkeiten}
Eine Möglichkeit, gute Transferleistung zu erreichen, ist das intensive einüben von Grundfertigkeiten, wie zum Beispiel in der Grundschule. \citet{LaBerge1974} untersuchten dies bei der Fertigkeit des Lesens. Dabei wird das Üben nicht abgebrochen, wenn die Schülerinnen und Schüler die Fertigkeit subjektiv bereits können, sondern noch einige Zeit fortgesetzt. \citeauthor{LaBerge1974} haben dabei Schülerinnen und Schüler einen Text solange laut vorlesen lassen, bis sie keinen Fehler mehr machten und einen hohen Flüssigkeitsgrad aufwiesen. Dieses \textit{überlernen} einer Fertigkeit fördert Transfer. So führt nach \citet{Perkins1989} hochgradig eingeübt Fertigkeiten zu spontanem automatischem Transfer, ohne dass es längeren Nachdenkens bedarf. Der Grund dafür liegt darin, dass Routinen gebildet wurden, welche in einer neuen Situation helfen, die Aufmerksamkeit verstärkt auf neue Aspekte zu richten \citep{LaBerge1974, Mietzel2007}.

Diese Erkenntnis deckt sich mit den Forderungen vom \citet{Whitehead1929}, welcher bereits 1929 davor warnte, dass in der Schule träges Wissen entsteht. Daher soll in der Schule darauf geachtet werden, nicht zu viel in zu kurzer Zeit zu erreichen. So fordertet er auch wenige Themen gebiete gründlich zu erarbeiten. Diese Forderung wurde auch von neueren Studien bestätigt \citep{Porter1989,Brophy1992a,Millar1999}. 

\subsubsection{Entkontextualiseren}

Wie bereits vorher angesprochen, hängt Wissen sehr stark vom Kontext ab unter welchem es gelernt wurde \citep{Godden1975,Schoenfeld1988}. \citet{Anderson1996} fordern daher, dass Wissen, so erworben werden soll, dass Lernende lernen, irrelevante Aspekte der Situation vom Wissensinhalt zu trennen. Dieser Prozess wird als Entkontextualiseren bezeichnet. Dadurch verliert der Lernende die Assoziation einer Aufgabe mit einem bestimmten Kontext und allmählich tritt das zugrunde liegende Prinzip hervor \citet{Perkins1989}. Entkontextualiseren von Wissen ist jedoch nicht ausreichend, zusätzlich müssen Lernende lernen, wann und wo welches Wissen angewendet werden muss \citet{Wiggins1993}. Diese Erkenntnis deckt sich mit den Ergebnissen von \citet{Gick1980}, bei denen die Teilnehmer einen höheren Transfer aufwiesen, wenn auf die Ähnlichkeit der Situationen hingewiesen wurden.


\subsubsection{Problemorientierter Unterricht}

\citet{Williams1992} untersuchte viele Lernsituationen im Medizinischen Studium auf ihre Möglichkeiten zu Transfer. Sie stellt fest, dass das Wissen, welches in Vorlesungen gelernt wurde, im klinischen Teil der Ausbildung vergessen ist. Ein Grund dafür sind, dass in vielen Lehrbüchern und Vorlesungen theoretisches Wissen losgelöst von Anwendungen dargestellt werden. So werden Fragen beantwortet, welche sich Lernende nicht stellen und daher von diesen nicht auf konkrete Problemsituationen angewendet werden können.
Diese Erkenntnis gilt nicht nur für Mediziner, sondern wurde auch in anderen Fachdisziplinen nachgewiesen. So bedauert \citet{Shuell1996}, dass zukünftige Lehrpersonen faktisches Wissen lernen, anstelle von anwendungsbezogenem Wissen. So lernen sie etwas \textit{über} das Unterrichten jedoch nichts darüber, \textit{wie} zu unterrichten ist.

Um diese Probleme zu vermeiden wurde problemorientierte Unterrichtsgelegenheiten entwickelt und untersucht (siehe unter anderem \citet{Barrows1985,Michael1993,Shuell1996,Corte2003,Reusser2005,Fassler2007,Pea2013b}). Das Ziel dabei ist, dass Wissen in möglichst lebensnahen Kontexten zu erwerben. Dies führt dazu, dass bei der Anwendung der Kontext ähnlich zu dem Kontext ist, unter welchem das Wissen erworben wurde.

\subsection{Zusammenfassung zu Transfer}

Es wurde in dem letzten Abschnitt versucht einen Überblick über der Begriff des Transfers zu geben und zu zeigen wie der Begriff des Kontextes damit verknüpft ist. Zuerst wurde eine historische Übersicht, über die wichtigsten Untersuchungen zu Transfer gegeben, um den Wandel des Begriffes des Transfers aufzuzeigen.  Als Vorbereitung für den nächsten Abschnitt wurde noch der Begriff des Transfers elementarisiert. Darauf Aufbauend wurden die Konsequenzen für den Unterricht, welcher transferierbares Wissen fördern soll zusammengefasst. Im nächsten Abschnitt geht es um den Begriff der Kompetenz und deren Verknüpfung mit dem Begriff des Transfers und Kontext.


\section{Kompetenz}

Nachdem ein Überblick über den Transfer erarbeitet wurde, soll in diesem Abschnitt versucht werden die Konsequenzen aus der Betrachtung zu Transfer mit dem Kompetenzbegriff zu verknüpfen.

\subsection{Bildungsreformen}
In den letzten Jahrzehnten fand international ein Wandel in der Bildungspolitik statt. In der Vergangenheit wurde der Fokus auf den Input des Bildungssystems gelegt. In den letzten Jahren fand eine Erweiterung der Perspektive statt und auch der Output des Bildungssystems wurde beachtet. Das Ziel dabei ist, die Qualität des Bildungssystems fassbar zu machen und soll helfen die Ressourcen effektiv einzusetzen.

Dieser Perspektivwechsel wurde von den grossen Bildungsstudien (PISA \citep{PISA-KonsortiumDeuschland2004}, TIMSS \citep{Martin2003} und IGLU \citep{Bos2003}) in den letzten Jahren ausgelöst. Diese führten zu einem Wandel, sowohl in der Forschung als auch in der politischen Diskussion über das Bildungssystem. So wurden die Resultate des Bildungssystems in den Vordergrund gerückt. Insbesondere die Definition von Standards und deren Verankerung im gesamten Bildungssystem sind neu. Davor wurden Standards meistens durch strukturelle Vorgaben umgesetzt (Lehrpläne, Stundentafeln und Schulorganisation). Diese Vorgaben haben einen Einfluss auf den Input des Bildungssystems. Die Qualität des Bildungssystems wurde jedoch nur sehr gering über die Überprüfung der erreichten Ergebnisse (Leistung der Schülerinnen und Schüler, Übertrittsquoten und Abschlussprüfungen) überprüft. Es wurde implizit angenommen, dass der Input einen Einfluss auf das Ergebnis des Bildungssystems als ganzes hat. 

Neu ist, dass die Steuerung des Bildungssystems vermehrt über den Output erfolgen soll. So soll die Leistung des Bildungssystems messbar gemacht werden und objektiv vergleichbar. Mit den bisherigen Leistungserhebungen auf Klassen oder Schulstufe, lässt sich der Output des Schulsystems nicht akkurat beschreiben, da festgelegte Messstandards fehlten. So wurden im Zuge der Entwicklung von Bildungsstandards kompetenzbezogene Niveaus eingeführt, welche einen Aufschluss über die erreichten Kompetenzen eines Schülers oder Schülerin geben soll \citep{Oelkers2008}.


Diese Bemühungen führten in vielen Ländern zur Entwicklung von neuen Bildungsstandards \citep{Berner2006}. In der Schweiz wurde dies von der \citet{EDKSchweizerKonfernezderKantonalenErziehungsdirektoren2004} unter dem Title "`Interkantonale Vereinbarung über die Harmonisierung der obligatorischen Schule (HarmoS-Konkordat)"' angestossen. In Deutschland wurde neue Bildungsstandards von der \citet{Kultusministerkonferenz2004} verabschiedet.

Im englischsprachigen Raum fanden diese Diskussionen bereits früher statt. So wurde in Neuseeland bereits zu Beginn der 1990er Jahren ein "`Outcome-based"' Curriculum verabschiedet \citep{McGee1996}. Auch in Australien wurde ein ähnliches Bildungskonzept 2000,  unter dem Name "`outcomes-based education (OBE)"', umgesetzt  \citep{Killen2000}. Auch in England wurde zu Beginn des Jahrtausends Bildungsreformen gefordert \citep{Millar1999}, welche dann um 2005 umgesetzt wurden \citep{Huber2006}.

%\subsection{Schweizer Bildungsstandard}



\subsection{Definition von Kompetenz}

Der Begriff der Kompetenz wird im Moment sowohl fachlich als auch politisch sehr stark diskutiert. So sprich \citet{Weinert2001b} von einer Inflation des Kompetenzbegriffes.
Der Begriff der Kompetenz, welcher in den Bildungsstandards (sowohl der Schweiz als auch von Deutschland) verwendet wird, basiert auf der Arbeit von \citet{Klieme2004}.
\citet{Klieme2004} unterscheidet verschiedene Varianten des Kompetenzbegriffes: 
\begin{enumerate}
\item Kompetenz als kognitive Leistungsdisposition, welche es Personen erlaubt unterschiedliche Aufgaben zu lösen.
\item Kompetenz als kontextspezifische kognitive Leistungsdisposition, welche sich auf spezifische Kontexte bezieht. Dieser Kompetenzbegriff wird oft mit Kenntnisse, Routinen oder Fertigkeiten bezeichnet.
\item Kompetenz als motivationaler Orientierungen, welche notwendig ist um eine Aufgabe zu bewältigen.
\item Handlungskompetenz, als Integration der vor-gängigen Kompetenzbegriffe, im Bezug auf die Anforderungen eines genau definierten Handlungskontextes.
\item Metakompetenzen als Strategiewissen oder Motivation, welche die Anwendung und den Erwerb anderer Kompetenzen erleichtert.
\item Schlüsselkompetenzen als generalisierbare kontextspezifische kognitive Leistungsdispositionen. Das heisst Kompetenzen, welche auf viele verschiedene Situationen angewendet werden können, wie zum Beispiel mathematische oder sprachliche Kenntnisse.
\end{enumerate}




\subsubsection*{Abgrenzung von Kompetenz und Intelligenz}
Problematisch an der Definition des Kompetenzbegriffes ist, dass der Kompetenzbegriff schwierig von der Definition der allgemeinen Intelligenz zu unterschieden ist, insbesondere der erste Punkt. \citet{Weinert2001b} empfiehlt daher eine Einschränkung des Kompetenzbegriffes. So sollen Kompetenzen auf einen eingeschränkten Raum von Kontexten und Situationen bezogen werden und allgemeine intellektuellen Fähigkeiten ausgeschlossen werden. Dies begründet \citet{Weinert2001b} damit, dass allgemeine intellektuelle Fähigkeiten eine Grundausstattung des Menschen sind und nicht erworben werden können und daher nur sehr begrenzt trainiert werden können. Zusätzlich schränkt er den Kompetenzbegriff weiter ein, indem er affektive und motivationale Aspekte nicht einbezieht. Der Begriff der Kompetenz soll daher auf spezifische Kenntnisse angewendet werden, welche notwendig sind, um genau definierte Ziele zu erreichen. Dies führt zu einer Abgrenzung zum Intelligenzkonzept, da damit Fähigkeiten assoziiert werden, welche ohne spezifisches Vorwissen auf neue Problemstellungen angewendet werden sollen. Daher ist der Begriff der Kompetenz stärker mit spezifischen Kontexten verbunden, während die Intelligenz sich generalisieren lässt. Dies führt jedoch zu einem weiteren Problem, da bei breiteren Kontexten die Abgrenzung zwischen Kompetenz und Intelligenz schwieriger wird \citep{Hartig2006}.


Ein weiterer Unterschied zwischen dem Kompetenz- und dem Intelligenzkonzept beruht auf der Lernbarkeit. So fordert \citet[S. 22]{Baumert2001}, dass Kompetenzen "`prinzipiell erlernbare, mehr oder minder bereichsspezifische Kenntnisse und Strategien"' sind. Dies bedeutet, dass Kompetenzen durch schulischen Unterricht gefördert und erweitert werden können. Daher sind dies Leistungen, welche durch den Schulbesuch verbessert werden sollten und sind daher für das Bildungsmonitorring von Interesse. Intelligenz wird hingegen als relativ stabil betrachtet, da Intelligenz hauptsächlich von genetischen Faktoren abhängt \citep{Shakeshaft2013}. Daher sollte theoretisch der Schulbesuch keine direkte Verbesserung der Intelligenzleistung zur Folge haben. Dies führt auch zu einem weiteren Unterschied zwischen der Intelligenz und der Kompetenz. So kann die Kompetenz in einem bestimmten Bereich bei null liegen, da die Erfahrungen, welche zu dem Erwerb der Kompetenz führen noch nicht gemacht wurden. Bei der Intelligenzleistung ist dies nicht möglich, da jeder Mensch sich diese Grundfertigkeiten irgendwo angeeignet haben sollte.


Des Weiteren gibt es bei Erstellung von Leistungsmessungen Unterschiede. Bei Intelligenztests werden bestimmte Primärfaktoren verwenden, z.B. dreidimensionales Denken, Gedächtnis, usw., welche Unterschiede zwischen einzelnen Personen aufzeigen sollen. Kompetenzen hingegen werden durch die Anforderungen definiert \citep{Rychen}. In anderen Worten, Kompetenzen werden durch die relevanten Aufgaben definiert, welche von den Untersuchten gelöst werden sollen. So werden in HarmoS Kompetenzen in einem dreidimensionalen Modell definiert, Themengebiete, Kompetenzaspekt und Kompetenzniveau \citep{KonsotriumHarmoSNaturwissenschaften+2010}. Die Kompetenzaspekte werden nicht wie bei der Intelligenz über psychischen Prozesse definiert, sondern aus spezifischen Anforderungen in spezifischen Kontexten abgeleitet. Diese Unterschiede führen dann auch zu einer unterschiedlichen Konstruktion von Leistungstests. Intelligenztests sollten so konstruiert sein, dass möglichst wenig Vorwissen für das Lösen von Aufgaben notwendig ist. Tests, welche auf Kompetenzen abzielen, wie z.B. PISA, wurden mit dem Ziel entwickelt, Aufgaben in realitätsnahen  Kontexten zu stellen. 

Trotz all dieser Unterschiede werden typischerweise hohe Korrelation zwischen Intelligenz- und Kompetenzleistungen festgestellt. So fand \citet{Rindermann2006} meistens eine sehr hohe Korrelation ($ > $0.7) zwischen Kompetenzen und anderen Massen für kognitive Fähigkeiten. Es ist aber mit den Resultaten von PISA nicht möglich festzustellen ob dies daran liegt, das Schüler und Schülerinnen eine höhere Kompetenz haben, weil sie intelligent sind, oder ob sowohl Intelligenz als auch Kompetenz durch die Schulbildung geprägt werden \citep{Hartig2006}.



\subsection{Kompetenz und Transfer}
Der Kompetenzbegriff wird in der Literatur sehr unterschiedlich definiert \citep{Klieme2004, Weinert2001b}. Dennoch ist die Definition des Kompetenzbegriffes für internationale Studien wie PISA \citep{PISA-KonsortiumDeuschland2004}, TIMSS \citep{Martin2003} und IGLU \citep{Bos2003} elementar.


Interessant ist, dass trotz der Einschränkung des Kompetenzbegriffes auf spezifische Kontexte, immer noch davon ausgegangen wird, dass die Kompetenz generalisierbar ist und teilweise auf andere Situationen übertragen werden kann \citet{Hartig2006}.

Im Abschnitt \ref{sec:TransferUnterricht} wurde herausgearbeitet, welche Eigenschaften von Unterricht zu besserer Transferleistung führen kann. Auch \citet{Lersch2007} gibt Vorschläge wie kompetenzfördernder Unterricht gestaltet sein sollte. So fordert er, dass der Unterricht "`viel stärker von den erforderlichen Lernprozessen und -gelegenheiten her konzipiert werden müsste und eben nicht nur von einer kontinuierlichen Abfolge von Inhalten"'. Dies deckt sich mit der Forderung von \citet{Mietzel2007} für das Entkontextualiseren von Unterricht um Transferleistung zu fördern. Auch die Problemorientierung von Lerngelegenheiten wird von \citet{Lersch2007} für kompetenzfördernden Unterricht als wichtig gehalten, insbesondere fordert er, dass "`systematische Wissensvermittlung […] um variable Anwendungssituationen"' ergänzt werden sollten. Zusätzlich fordert er, dass realistische Lernsituationen angeboten werden sollten, in anderen Worten: die Lerngelegenheiten sollten mit dem Ziel der Kompetenz übereinstimmen, da der Erwerb der Kompetenz ja kontextspezifische erfolgt \citep{Klieme2004}.



\section{Kompetenz des skalenbasiertes Messens}

Im Rahmen von ExKoNawi hands-on Experimentiertests wurde ein Modell entwickelt um verschiedene hands-on Kompetenzen von Schülerinnen und Schüler auf der Sekundarstufe I in der Schweiz zu messen werden \citep{Metzger2013}. Einer der Kompetenzen welche mit ExKoNawi hands-on Experimentiertests gemessen werden soll, ist die Kompetenz des \textit{skalenbasiertes Messen} \citep{Gut2013a}. Die Definition dieser Kompetenz basiert auf der Arbeit von \citet{Munier2013}. In dieser Kompetenz geht nach \citep{Gut2013a} darum "`quantitative Grössen mit gegebenen Messinstrument genau [zu] messen"'. Bei dieser Kompetenz gibt es drei Teilbereiche die eine wichtige Rolle spielen. Zum einen müssen die Schülerinnen und Schüler entscheiden, welches Messinstrument besser für eine Messung geeignet ist. Ein weiterer Teilaspekt ist, dass sie die Messung mehrmals wiederholen um eine genauere Abschätzung des Resultates bekommen. Zusätzlich zu diesen Aspekten müssen sie auch das Messinstrument korrekt verwenden \citep{Munier2013,Gut2013a}.

Ein wichtiger Aspekt der Kompetenz des skalenbasierten Messens ist, dass diese Kompetenz ohne einen inhaltlichen oder fachlichen Kontext definiert wurde. Was bedeuten sollte, dass das Kompetenzniveau, welches ein Schüler oder eine Schülerin erreichen könnte unabhängig des fachlichen oder inhaltlichen Kontextes sein müsste, in welcher die Messung der Kompetenz stattfinden sollte.

\section{Forschungsfrage}

Dieses Modell führt nun daher zur Frage, ist das erreichbare Kompetenzniveau von Schülerinnen und Schüler tatsächlich unabhängig des inhaltlichen und fachlichen Kontextes? Insbesondere auch aus dem Aspekt, dass sich kontextspezifische Kompetenzen und Transferleistungen grundsätzlich nicht gegenseitig ausschliessen. Guter komptenzorientierter Unterricht unterstützt hingegen sogar die Fähigkeiten das Wissen zu transferieren. Diese Erkenntnis führt nun jedoch zu der Frage: 
\begin{quote}
Ist eine gewisse Kompetenz (hier skalenbasiertes Messen) von Lernenden auf der Sekundarstufe I in unterschiedlichen Kontexten gleich verfügbar?
\end{quote}

Diese Frage verknüpft sehr stark den Begriff des Transfers mit dem Kompetenzbegriff. Im
Bezug auf den Transferbegriff müssen Lernende eine Transferleistung erbringen, da sie
diese gewisse Kompetenz auf verschiedene Kontexte anwenden müssen. Um diese Frage zu beantworten soll in der vorliegenden Arbeit untersucht werden ob die erreichten Kompetenzniveaus des skalenbasierten Messens unabhängig des fachlichen oder inhaltlichen Kontextes sind.



