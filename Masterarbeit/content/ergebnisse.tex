\section{Kodierung}

Wie bereits geschrieben wurde die Erstkodierung von David Sichau durchgeführt. Es wurden eine Zweitkodierung von  15 \% zufällig ausgewählten (per Random Generator) Auswertungsbögen von Pitt Hild  durchgeführt. Es wurden dabei die gleichen Kodierschemata verwendet, welche sich im Anhang der Arbeit befinden. %TODO Referenz zu Koderischemata

\subsection{Items}
Es gab insgesamt elf Items welche nach dem Kodierschemata kodiert wurden.

Die Items wurden auf Interrater-Reliabilität untersucht. Dafür wurden die prozedurale Übereinstimmung $p_0$ und zusätzlich noch das ungewichtete Cohen's Kappa $\kappa$ als Zufalls-korrigierter Koeffizient berechnet. Bei einem Teil der Datensätze war dies mathematisch nicht möglich (Division durch 0), daher können nicht für alle Items ein Cohen's Kappa angegeben werden. In Tabelle \ref{tab:CohenKappa} sind alle Ergebnisse zusammengefasst.


\github{http://git.io/mk9z-Q}

\begin{table}[htbp]
  \centering
\begin{tabular}{ccccccc}
\toprule   &  \multicolumn{2}{c}{201} &  \multicolumn{2}{c}{301}  & \multicolumn{2}{c}{301}\\
Item  & $p_0$ & $\kappa$ &  $p_0$ & $\kappa$ &  $p_0$ & $\kappa$\\
\midrule
 1.1 & 1 & 1  & 0.91 & 0.74 & 0.91 & 0.79 \\ 
 1.2 & 0.91 & 0.81  & 1 & /  & 1 & 1 \\ 
 2.1 & 0.81 & 0.67  & 0.81 & 0.74  & 1 & 1\\ 
 3.1 & 1 & 1  & 0.91 & 0.81  & 1 & 1\\ 
 3.2 & 1 & /  & 1 & 1  & 0.91 & 0.82\\ 
 4.1 & 0.91 & 0.79 & 0.81 & 0.65  & 0.91 & 0.81 \\ 
 4.2 & 0.91 & 0.62 & 0.91 & 0.79  & 0.91 & 0.74 \\ 
 4.3 & 1 & /  & 1 & /  & 1 & / \\ 
 4.4 & 1 & /  & 1 & /  & 1 & / \\ 
 5.1 & 1 & /  & 1 & /  & 1 & / \\ 
 5.2 & 0.91 & /  & 1 & 1  & 0.91 & 0.78 \\ 

\bottomrule

\end{tabular} 
  \caption{Übereinstimmung der Kodierungen für die einzelnen Items ($p_0$) und Cohens Kappa $\kappa$. Für die drei Tests 201 (Chemie-Temperatur), 301 (Physik Kraft) und 305 (Physik Temperatur)}
  \label{tab:CohenKappa}
\end{table}

\subsection{Qualitätsstandards}
Aus den elf Items wurden fünf Qualitätsstandards entwickelt \citep{Hild2014a}.
\subsubsection*{Qualitätsstandard 1}
Im Qualitätsstandard 1 geht es um das korrekte und präzise messen. Dieser Qualitätsstandard wird nur erreicht wenn Item 1.1 (Richtige Tendenz des Resultates) und Item 1.2 (Ist das Resultat vollständig und korrekt) zusammen mindestens 1 ergeben.


\subsubsection*{Qualitätsstandard 2}
Bei Qualitätsstandard 2 wird die Dokumentation der Messung bewertet . Dieser Qualitätsstandard wird nur erreicht wenn Item 1.2 (Werden alle Messungen und Messergebnisse vollständig dargestellt) mindestens den Wert von 2 erreicht hat. 

\subsubsection*{Qualitätsstandard 3}
Im dritten Qualitätsstandard wird das Begründen des richtigen Messinstrumentes bewertet. Dieser Standard wird nur erreicht wenn Item 3.1 (Ist das Korrekte Messinstrument gewählt worden) und Item 3.2 (Wird die Wahl des Messinstrumentes korrekt begründet) zusammen zwei ergeben.

\subsubsection*{Qualitätsstandard 4}
Qualitätsstandard 4 beurteilt die Messwiederholung. Es wird aus Item 4.1 (mehrmaliges Messen), 4.2 (identische Messung), 4.3 (wurde Mittelwert gebildet) und 4.4 (korrekter Mittelwert) gebildet. Diese Level wird erreicht wenn die Items addiert mindestens zwei ergeben.

\subsubsection*{Qualitätsstandard 5}
Der letzte Qualitätsstandard 5 zeigt auf, inwiefern die Schülerinnen und Schüler Fehlerquellen der Messung begründen können. Dieser Standard besteht aus Item 5.1 (Fehlerkategorien nennen) und 5.2 (Verbesserungsvorschläge) welche zusammen mehr als eins ergeben müssen.

\subsubsection{Erreichte Qualitätsstandards}

In Tabelle \ref{tab:QS} wird ein Überblick über die erreichten Qualitätsstandards aller Schülerinnen und Schüler gegeben. Zusätzlich werden auch die bedingten Qualitätsstandards angeben, welche nur erreicht werden können, wenn der vorhergehende Qualitätsstandard erreicht wurde.


\begin{table}[!htbp]
  \centering
\begin{tabular}{ccccccccccc}
\toprule
 Test & $p_{Q1}$ & $p_{QS1}$ & $p_{Q2}$ & $p_{QS2}$& $p_{Q3}$& $p_{QS3}$& $p_{Q4}$& $p_{QS4}$& $p_{Q5}$& $p_{QS5}$\\ 
\midrule
 201 &   0.51 & 0.51& 0.34 & 0.27 & 0.05 & 0.04 & 0.08 & 0.03 & 0.16 & 0.03 \\ 
 301 &   0.62 & 0.62& 0.31 & 0.31 & 0.09 & 0.04 & 0.09 & 0.01 & 0.39 & 0.01\\ 
 305 &   0.72 & 0.72& 0.30 & 0.29 & 0.35 & 0.14 & 0.11 & 0.01 & 0.50 & 0.01\\ 
\bottomrule
 
\end{tabular} 

  \caption{Zusammenfassung der erreichten Qualitätsstandards, wobei $p_{Q1} - p_{Q5}$ den unbedingten Qualitätsstandards entsprechen. Die bedingten Qualitätsstandards werden mit $p_{QS1} - p_{QS5}$ bezeichnet.}
  \label{tab:QS}
\end{table}

\subsection{Niveau}

Basierend auf den Qualitätsstandards wurden zwei Niveaus gebildet, welche das erreichte Niveau der Schülerinnen und Schüler bei der Kompetenz des skalenbasierten Messens bezeichnen. Die Niveaus können einen Wert zwischen 0 und 5 annehmen. Eine Übersicht über die erreichten Niveaus wird in Tabelle \ref{tab:Niveau} gegeben.
\github{http://git.io/bjn9qg}
\begin{table}[!htbp]
  \centering
\begin{tabular}{c|cccccc|cccccc}
\toprule

  \multicolumn{1}{c}{} &  \multicolumn{6}{c}{uLev} &  \multicolumn{6}{c}{cLev}\\ 
 Test & 0 & 1 & 2 & 3 & 4 & 5 & 0 & 1 & 2 & 3 & 4 & 5\\ 
\midrule
 201 &   0.36 & 0.24 & 0.22 & 0.13 & 0.03 & 0.03 & 0.40 & 0.24 & 0.32  & 0.01 & 0 & 0.03   \\ 
 301 &   0.31 & 0.21 & 0.29 & 0.14 & 0.03 & 0.03  & 0.42 & 0.28 & 0.26 & 0.01 & 0 & 0.03  \\ 
 305 &   0.13 & 0.19 & 0.24 & 0.31 & 0.11 & 0.03  & 0.22 & 0.43 & 0.18 & 0.13 & 0.01 & 0.03 \\ 
\bottomrule
 
\end{tabular} 

  \caption{Prozedural erreichte Niveaus aller Schülerinnen und Schüler. Das unbedingte Niveau wird mit uLev und das bedingte Niveau mit cLev bezeichnet. }
  \label{tab:Niveau}
\end{table}

\subsubsection{Unbedingtes Niveau}
Dieses Niveau ist der Summenscore der einzelnen unbedingten Qualitätsstandards. In der Arbeit wird dieses Level mit \textit{uLev} abgekürzt.

\subsubsection{Bedingtes Niveau}

Dieses Niveau ist der Summenscore der bedingten Qualitätsstandards. Dieses Niveau wird mit \textit{cLev} abgekürzt.




\section{Fragebogen}

Im standardisierten Teil des Fragebogens wurden Fragen zum absoluten Selbstkonzept nach SESSKO gestellt \citep{Schone2002}. Die verwendeten Fragen sind in Tabelle \ref{tab:SESSKO} aufgeführt. 



\begin{table}[htbp]
  \centering
\begin{tabular}{|p{3cm}|p{9cm}|p{1cm}|}
\hline Skala & Frage & $\alpha_d$  \\ 
\hline SESSKO 18(a) & Ich bin für die Schule sehr begabt. &  0.71  \\ 
\hline SESSKO 19(a) & Neues zu lernen fällt mir schwer.  &  0.76 \\ 
\hline SESSKO 20(a) & Ich bin sehr intelligent. &  0.71  \\ 
\hline SESSKO 21(a) & Ich kann in der Schule viel. &  0.72   \\ 
\hline SESSKO 22(a) & In der Schule fallen mir viele Aufgaben schwer.  & 0.74   \\ 
\hline 
\end{tabular} 

  \caption{Fragen von SESSKO zur Skala "`Schulisches Selbstkonzept - absolut"'  \citep{Schone2002}. $\alpha_d$ bezeichnete das standardisierte Cronbach Alpha wenn dieses Item weggelassen würde.}
  \label{tab:SESSKO}
\end{table}

Zusätzlich wurden nach \citet{Dierks2014} Fragen zum Selbstkonzept zu Schulversuchen entwickelt und angepasst. Die entwickelten Fragen sind in Tabelle \ref{tab:NatSK} aufgezeigt.

\begin{table}[htbp]
  \centering
\begin{tabular}{|p{2cm}|p{10cm}|p{1cm}|}
\hline Kürzel & Frage & $\alpha_d$  \\ 
\hline NatSK1 & Schulversuche liegen mir nicht besonders. &  0.65  \\ 
\hline NatSK2 & Schulversuche würde ich viel lieber machen, wenn sie nicht so schwer wären.  &  0.69 \\ 
\hline NatSK3 & Schulversuche fallen mir schwerer als vielen meiner Mitschüler/innen. &  0.65  \\ 
\hline NatSK4 & Bei manchen Schulversuche weiss ich gleich: "`Das verstehe ich nie."' &  0.65   \\ 
\hline NatSK5 & Für Schulversuche habe ich einfach keine Begabung.   & 0.63   \\ 
\hline NatSK6 & Mit den Aufgaben bei Schulversuche komme ich besser zurecht als viele meiner Mitschüler/innen  & 0.67   \\ 
\hline NatSK7 & Ich denke, ich bin für Schulversuche begabter als viele meiner Mitschüler/innen.  & 0.66   \\ 
\hline 
\end{tabular} 

  \caption{Fragen zum Sebstkonzept bei Schulversuchen abgewandelt nach \citet{Dierks2014}. $\alpha_d$ bezeichnete das standardisierte Cronbach Alpha wenn dieses Item weggelassen würde.}
  \label{tab:NatSK}
\end{table}

Es wurde die innere Konsistenz beider Skala überprüft. Bei den der Skala "`Schulisches Selbstkonzept - absolut"' wurde ein standardisiertes Cronbach $\alpha$ von 0.77 erreicht. Die Anzahl vollständig ausgefüllter Fragebögen betrug dabei 69. Alle unvollständigen Items wurden vor der Analyse entfernt. Bei der Skala zum Selbstkonzept bei Schulversuchen wurde ein standardisiertes Cronbach $\alpha$ von 0.69 erreicht. Insgesamt konnten dabei 64 vollständige Fragebögen ausgefüllt werden. 
\github{http://git.io/WyJH6Q}


\section{Unterschiede zwischen den Klassen}

Um festzustellen, ob alle Datensätze der einzelnen Klassen kombiniert werden dürfen wurden zuerst alle Klassen einzeln gegeneinander auf folgende Nullhypothese überprüft: Besteht \underline{kein} Unterschied in den Qualitätsstandards zwischen den einzelnen Klassen?

Aufgrund der geringen Anzahl an Datenätzen in den einzelnen Klassen wurde der exakter Test nach Fisher verwendet. Dieser Test wurde für jeden Qualitätsstandard (bedingt und unbedingt) einzeln jeder Klasse gegen jede andere Klasse durchgeführt. Bei keinem der 60 Tests konnte die Nullhypothese abgelehnt werden (p<0.05). Daher gibt es keinen signifikanten Unterschied zwischen den erreichten Qualitätsstandards und den einzelnen Klassen.  

\github{http://git.io/0DOelQ}




