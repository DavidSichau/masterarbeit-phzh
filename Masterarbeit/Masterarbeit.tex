\documentclass[12pt,oneside, DIV11]{scrbook}

% deutsche Silbentrennung

\usepackage[ngerman]{babel}

\usepackage{davidLayout}

\usepackage{float}

\usepackage[
	natbib,
	bibencoding=utf8,
	isbn=false,
	backend=biber,
	doi=false,
	url=false,
	authordate
]{biblatex-chicago}  
\DefineBibliographyStrings{german}{%
  andothers = {et al.},
}
%do not short same author names with dash
\renewbibmacro*{author}{%
  \ifboolexpr{
    test \ifuseauthor
    and
    not test {\ifnameundef{author}}
  }
    {\printnames{author}%
        \iffieldundef{authortype}
          {\setunit{\addspace}}
          {\setunit{\addcomma\space}}%
     \iffieldundef{authortype}
       {}
       {\usebibmacro{authorstrg}%
        \setunit{\addspace}}}%
    {\global\undef\bbx@lasthash
     \usebibmacro{labeltitle}%
     \setunit*{\addspace}}%
  \usebibmacro{date}}

\addbibresource{../../../../library.bib}

\usepackage{booktabs} %make nicer tables
\renewcommand{\arraystretch}{1.2} % more space on rows in tables (or 1.3)



\usepackage{xcolor}
\usepackage{hyperref}
\hypersetup{
	colorlinks=true,
	linkcolor=blue,
	urlcolor=blue,
	citecolor=blue
}

\usepackage[framemethod=TikZ]{mdframed}
\mdfsetup{%
	leftmargin =+5cm,
	rightmargin=+5cm,
   	roundcorner=5pt,
   	backgroundcolor=gray!5,
   	nobreak=true,
   	usetwoside=false
}

\newcommand{\github}[1] {
  \begin{mdframed}
  \begin{center}
  Code erhältlich auf:\\
  \href{#1}{
    \includegraphics[width=0.3\textwidth]{graphics/GitHubLogo.png}
    }\\
    \vspace{-0.3cm}
    {\tiny \href{#1}{#1}}
  \end{center}
  
  \end{mdframed}
}

\usepackage[final]{pdfpages}


\begin{document}
\frontmatter

\begin{titlepage}
	\vspace*{2cm}
	\begin{center}
		{\LARGE Vom Einfluss des Kontextes auf Kompetenzen im Rahmen von hands-on Experimentiertests
		 \vspace*{2cm}\\ Masterarbeit an der Pädagogischen Hochschule Zürich\vspace*{1cm}\\Masterstudiengang Fachdidaktik Naturwissenschaften\\}
		\vspace*{2cm}{\normalsize vorgelegt von:}\\ \large David-Matthias Sichau \\
		\vspace*{1.5cm} {\normalsize  eingereicht bei:}\\ \large Pitt Hild \\
		\vspace*{2cm}{\large 05. Februar 2015, Zürich}\\
	\end{center}
\end{titlepage}



\frontmatter 
\tableofcontents



\chapter*{Zusammenfassung}

In der vorliegenden Masterarbeit wurde untersucht, ob die Kompetenz des skalenbasierten Messens von Lernenden auf der Sekundarstufe I in unterschiedlichen Kontexten gleich verfügbar ist. Dafür wurden hands-on Experimentiertests im Rahmen des ExKoNawi Projektes der PH Zürich verwendet, bei denen die gleiche Kompetenz in unterschiedlichen Kontexten gemessen wird, verwendet. In Rahmen dieser Arbeit konnte gezeigt werden, dass für die hier untersuchte Stichprobe (72 1. Sek A Schülerinnen und Schüler) die Kompetenz des skalenbasierten Messens unabhängig vom fachlichen oder inhaltlichen Kontext ist.





\mainmatter


\chapter{Einleitung}

\chapter{Einleitung}

Im Rahmen der ExKoNawi hands-on Testaufgaben soll untersucht werden inwiefern die Qualitätsstufen von verschiedenen Test zur Erhebung der Kompetenz des Messens korrelieren. 

Dabei soll insbesondere untersucht werden, ob der fachliche Kontext oder der inhaltliche Kontext ausschlaggebend ist. In anderen Wort hängt die Leistung der SuS davon ab, ob die Kompetenz des Messens im gleichen Fach gemessen wird, oder hängt dies davon ab was für eine Messung durchgeführt wird.

Diese Frage ist besonders von Interesse, wenn es darum geht Test auszuwählen. Ist es möglich die Kompetenz des Messens mit einem einzelnen Test zu erheben oder sind mehrerer Tests notwendig um das erreichte Kompetenz Niveau ausreichend zu beschreiben.







\chapter{Theoretischer Rahmen}
\chapter{Theorie und Problemlage}

\section{Transfer}

Es wird erwartet, dass in der Schule vermitteltes Wissen universell aufgerufen werden kann und auch das Gelernte auf andere Kontexte angewendet werden kann. Dieses universell verfügbare Wissen ist eng mit dem Begriff des Transfers verknüpft. \citet{Greeno1996} definierten Transfer als "`the process of applying knowlege in new situations"'. Aber auch innerhalb der schulischen Bildung gibt es einen Transfer zwischen den verschiedenen Fächern. So wird von SuS erwartet, dass Fähigkeiten, Lösungsstrategien, Konzepte und anderes Wissen auf andere Fächer übertragen werden soll und dort abgerufen werden kann.

\subsection{Historischer Überblick}


\subsubsection{Woodworth 1901}

Eines der ersten Experimente zu Transfer wurde von \citet{Woodworth1901} gemacht. Dabei mussten Probanden die Grösse von Rechtecken schätzen. Nachdem die Personen durch Wiederholungen sich verbessert hatten wurde ihnen zwei neue Test Sets gegeben. In einem gab es neue Rechtecke, welche im ursprünglichen Set nicht enthalten waren. Die zweite Gruppe bekam Sets bei denen andere Formen enthalten waren (z. B. Kreise und Dreiecke). Die zweite Testgruppe machte ähnlich viel Fehler, wie vor dem Training mit den Rechtecken. Daraus schloss \citeauthor{Woodworth1901}, dass kein Transfer stattgefunden haben kann.
Ein Beispiel für eine Untersuchung auf universitärem Niveau ist \citet{Renkl1994}. Er konnte zeigen, dass Nichtökonomen eine simulierte Firma besser führten, als Studierende der Betriebswissenschaften kurz vor Ihrem Abschluss. Diese Resultate führen zu dem Schluss, dass Transfer nur sehr schwierig erreicht werden kann, und wenn oft nur unter sehr ähnlichen Bedingungen.
Dieses Transfers Verständnis basiert und stützt das Reiz-Reaktions-Modell des Lernens \citep{Detterman1993, Mietzel2007}.
\subsubsection{Ferguson 1956}
Eine alternative Theorie zum Transfer wurde von \citet{Ferguson1956} entwickelt. Fergusons Theorie basiert darauf, dass die Intelligenz einer Person sich auf deren Transferleistung auswirkt. So findet nach \citet{Ferguson1956} bei dem Lernen permanent ein Transfer statt, da jede Lernaufgabe von der anderen unterschiedlich ist und daher Transfer stattfinden muss. Im Unterschied zu \citet{Woodworth1901} betrachtet \citeauthor{Ferguson1956} Transfer als einen kontinuierlichen Prozess, welcher durch Lernen verbessert werden kann. Wichtig ist jedoch zu beachten, dass \citeauthor{Ferguson1956} Theorie nur Nah-Transfer beschreibt. Unter Nah-Transfer wird Transfer zwischen sehr ähnlichen Situationen definiert. 
\subsubsection{Judd 1908}
Eine der grundlegenden Studien zu Fern-Transfer, bei welchem erworbenes Wissen auf Kontexte angewendet werde soll, welche sich deutlich vom Kontext, unter welchem das Wissen erworben wurde, unterschieden, wurde von \citet{judd1908} gemacht. Im Vergleich zu \citeauthor{Woodworth1901} geht Judd davon aus das der Unterschied zwischen den beiden Situationen nicht nur abhängig von der Ähnlichkeit un den Unterschieden zwischen den beiden Situationen ist, sondern auch davon abhängt wie die erste Situation gelernt wurde. Um dies zu belegen führte Judd eine sehr bekannte Studie durch. Bei dieser wurden Kinder genommen, welche mit einem Dart auf eine Zielscheibe unter Wasser werfen sollten. Beide Gruppen bekamen zu beginn die Möglichkeit dies zu trainieren. Später wurde das Werfen wiederholt, wobei die Postition der Zielscheibe jedoch unterschiedlich war, und untersucht welche Gruppe besser war. Eine der beiden trainierten Gruppen wurde währen sie die Situation A trainierten erklärt, warum die Scheibe so schwierig zu treffen war. Ihnen wurde also das Prinzip der Lichtbrechung erklärt. Die Gruppe welche die Erklärung bekommen hatte schnitt unter der neuen Situation deutlich besser ab, als die andere Gruppe. \citet{judd1908} erklärte dieses damit, dass die einen wussten, welches Prinzip sie auch bei der zweiten Situation anwenden können. Denen das Prinzip nicht erklärt wurden, haben gelernt ihren Wurf auf die erste Situation anzuwenden, konnten dieses Wissen jedoch nicht generalisieren, da dies spezifisch für die Situation erworben wurde. Im Vergleich zu \citeauthor{Woodworth1901} beinhaltet die Theorie von \citeauthor{judd1908} einen kognitivistisches Verständnis des Lernens. Da die Lernenden ein immer besseres Verständnis der Welt um sich selbst konstruieren und so neue Situation basierend auf ihrer internen Repräsentation der Welt lösen können. \citet{Detterman1993} kritisiert an dieser Studie die Verwendung von Transfer. So erklärt \citeauthor{judd1908} einem Teil der Personen das zugrunde liegende Prinzip. Dies ist laut \citeauthor{Detterman1993} jedoch so, wie wenn man den Personen sagen würde, dass sie dieses Prinzip verwenden sollen. Was dann identisch wäre, wie wenn man einer Anleitung folgen würde.

\subsection{Gick und Holyoak 1980}

Eine weitere bedeutende Studie zu Transfer wurde von \citet{Gick1980} durchgeführt. Dabei wurde untersucht, unter welchen Bedingungen Lernende Analogien verwenden um strukturell ähnliche Probleme zu lösen. Ein Beispiel Problem welches sie den Lernenden gaben war, wie kann ein Tumor mit Strahlung zerstört werden ohne dass gesundes Gewebe geschädigt wird. Dieses Problem wurde erstmals von \citet{Duncker1945} verwendet. Dieses Problem kann gelöst werden, indem man mehrere Strahlen verwendet, welche sie nur im Tumor überlagern. Bevor sie dieses Problem lösten bekommen Sie eine Geschichte erzählt, bei welcher das gleiche Prinzip verwendet wird. In dieser Geschichte ging es darum ein Fort das von Minen umgeben ist zu erobern. Durch aufteilen der Angreifer in mehrere angreifende Gruppen, die unterschiedliche Wege gehen, wurde die Belastung auf die Minen reduziert und das Fort konnte erobert werden. Das Resultat dieser Studie zeigte, dass spontaner Transfer nur sehr selten stattfindet. Das Hören der Geschichte führt nicht zu einer höheren Wahrscheinlichkeit das zweite ähnliche Problem zu lösen, solange die Lernenden nicht auf die Ähnlichkeit aufmerksam gemacht werden.

\subsection{Kritiken}

\citet{Lave1988} kritisiert die verschieden hier vorgestellten Untersuchungen. Da bei allen angenommen wird, das  Wissen automatisch generalisierbares Wissen erzeugt, welches auf verschiedene Situationen angewendet werden kann. Sie schlägt eine Alternative vor welche sie als "'practice view"' bezeichnet. Bei dieser wird Wissen von Personen erworben, welche an speziellen Übungen teilnehmen und daraus nur Wissen entwickelt wird, welches auf diese spezifische Situation zutrifft.

Folgende Kritiken erhebt sie. So stellt sie die Frage was lernen die Teilnehmer der verschiedenen Studien überhaupt. So greift sie insbesondere die Annahme an, dass die Teilnehmer der Studien Kontext unabhängig lernen. Sie lernen immer Kontext spezifisch. Ein anderer Punkt welche Sie angreift ist, wer definiert die Ähnlichkeit der Probleme. Ist die Ähnlichkeit der Probleme für die Teilnehmer der Studie auch greifbar. Auch \citet{Detterman1993} kritisiert die Studien. So sollten seiner Meinung alle Studien zu Transfer als Doppel-Blind Studien durchgeführt werden, da der Studienleiter unbewusst die Leistung der Probanden ändern kann. \citet{Detterman1993} fordert daher:
\begin{quote}
No tranfer experiment should be carried out without using a double blind procedure, particularly experiments assessing general transfer \citet[S. 10]{Detterman1993}.
\end{quote}

\subsection{Definition von Transfer}

Nachdem einige grundlegenden Studien zu Transfer exemplarisch aufgezeigt wurden, soll nun der Begriff des Transfers genauer definiert werden. 


\citet{Lobato2002a} möchte einen Kritik-Punkt von \citet{Lave1988} lösen. So kritisierte \citeauthor{Lave1988}, das der Untersucher festlegt was Transfer ist. So legten sie als Messung für den Transfer fest, welche Ähnlichkeit der Proband selbst zwischen verschiedenen Situationen zieht. So untersuchten Sie einen Schüler, welcher eine Rollstuhlrampe erhöhen sollte ohne die Steigung zu verändern. Der Schüler löste dieses Problem in dem er die Verhältnisse von Höhe zu Länge konstant hielt. Er verwendete nicht die im Mathematik Unterricht gelernten Formeln für dieses Problem. In den bisherigen Untersuchungen wäre daher angenommen, dass der Schüler keinen Transfer geleistet hat. Aufgrund der Interviews stellte sie jedoch fest, dass der Schüler sehr wohl Transfer geleistet hat, indem er das Konzept von konstanter Geschwindigkeit als Verhältnis von zurückgelegter Strecke zur Zeit auf dieses Problem angewendet hat.

\citet{Lobato2002a} konnten damit zeigen, dass wenn man die Ähnlichkeit zwischen zwei verschiedenen Situationen nicht mit strukturellen Ähnlichkeiten oder Unterschieden beschrieben werden sollen. Sondern damit, wie der Lernende die Ähnlichkeiten zwischen den Situationen wahr nimmt.

\subsection{Elemente von Transfer}

Nachdem ein Überblick über die historische Entwicklung von Transfer gegeben wurde und auch die aktuelle Definition diskutiert wurde soll nur auf die grundlegenden Elemente welche bei Transfer anzutreffen sind eingegangen werden.

\citet{Marini1995} definiert drei Elemente, welche zu einem Transfer führen. Das erste Element sind Merkmale des Lernenden. Dieser hat sobald er eine Situation antrifft bereits ein bestimmtes prozedurales und deklaratives Wissen, welches er sich erarbeitet hat und abrufen kann. In einem bestimmten Kontext kann er einen teil davon abrufen und anwenden \citep[s. S. 189ff]{Marini1995}. Dies führt dazu das lösungsrelevantes Wissen von den Vorhanden und dem verarbeitbarem Wissen abhängt. Zusätzlich kann in einem bestimmten Kontext jedoch nicht alles Wissen abrufen werden, da man mit trägem Wissen rechnen muss und auch der aktuellen Motivation des Lernenden. 

Als zweites Element von Transfer gibt \citeauthor{Marini1995} die Merkmale einer Aufgabenstellung an. So hängt Transfer von der Ähnlichkeit der Aufgabe ab. Dabei gibt es jedoch einen Unterschied zwischen Novizen und Experten. Novizen vergleichen Aufgaben hauptsächlich aufgrund oberflächlicher Merkmale, wohingegen Experten sich auf die zugrunde liegenden Prinzipien fokussieren \citep[s. S. 279]{Marini1995}. Aufgrund dessen, haben Novizen oft Problem den Zusammenhang zwischen Aufgaben zu sehen und können daher keinen Transfer durchführen.

Das dritte Element ist der Kontext in den ein Problem eingebettet ist. Ein Beispiel dafür ist die Untersuchung von \citet{Godden1975}. Dort lernten Taucher Wörter Unterwasser auswendig. Bei einer späteren Überprüfung konnten sie sich an mehr Wörter erinnern, wenn es Unterwasser wiederholt wurde im Vergleich zu einer Wiederholung auf dem Festland. Dieser Ortswechsel ist auch bei ausserschulischem Kontext gegeben. Aber auch innerhalb der Schule kann es zu unterschieden kommen. Ein Beispiel dafür liefert \citet{Schoenfeld1988} so hatten Lernende keine Schwierigkeiten mit einer Divisionsaufgabe. Wenn die Aufgabe jedoch in einen Kontext gestellt wurde, in eine Textaufgabe eingebettet, scheiterten die meisten der Lernenden. 

Erst durch die Berücksichtigung aller drei Elemente lässt sich Transfer ganzheitlich Betrachten. Nicht wie  \citet{Woodworth1901}, welcher nur den Aspekt der Aufgabenmerkmale genauer untersucht hat. Erst neuere Arbeiten berücksichtigen alle Elemente und insbesondere den Kontext \citep{Lobato2002a, Detterman1993, Greeno1996}

\subsection{Konsequenzen für den Unterricht}

Wie \citet{claxton1990} zeigte, darf jedoch nicht davon ausgegangen werden, dass in der Schule erworbenes Wissen ohne weiteres auf andere Alltags Probleme angewendet werden können. So sprach \citet{Whitehead1929} von "`trägem Wissen"' (inert knowledge), wenn Wissen vorhanden ist um ein Problem zu lösen, dieses jedoch nicht automatisch abgerufen werden kann. Dieses Wissen ist erst greifbar, wenn die Person angeregt wird dieses Wissen zu verwenden. Nach \citet{Whitehead1929} entsteht träges Wissen oft unter schulischen oder universitären Bedingungen. 

Wie muss nun schulischer Unterricht aussehen, welcher verhindert, das träges Wissen entsteht und möglichst viel Transfer von Wissen stattfinden kann. \citet[s. S. 316ff]{Mietzel2007} schlägt verschiedene Strategien vor wie dies verhindert werden kann.

\section{Kompetenz}

Nachdem ein Überblick über den Transfer erarbeitet wurde, soll in diesem Abschnitt versucht werden die Konsequenzen aus Transfer mit dem Kompetenzbegriff zu verknüpfen.


\chapter{Methode}



Der neue Test wurde im Rahmen einer Masterarbeit parallel evaluiert. Für eine detaillierte Beschreibung der Umsetzung sei auf die Masterarbeit verweisen \citep{Sichau2015a}.

\chapter{Ergebnisse}
\section{Kodierung}

Wie bereits geschrieben wurde die Erstkodierung von David Sichau durchgeführt. Es wurden eine Zweitkodierung von  15 \% zufällig ausgewählten (per Random Generator) Auswertungsbögen von Pitt Hild  durchgeführt. Es wurden dabei die gleichen Kodierschemata verwendet, welche sich im Anhang der Arbeit befinden. %TODO Referenz zu Koderischemata

\subsection{Items}
Es gab insgesamt elf Items welche nach dem Kodierschemata kodiert wurden.

Die Items wurden auf Interrater-Reliabilität untersucht. Dafür wurden die prozedurale Übereinstimmung $p_0$ und zusätzlich noch das ungewichtete Cohen's Kappa $\kappa$ als Zufalls-korrigierter Koeffizient berechnet. Bei einem Teil der Datensätze war dies mathematisch nicht möglich (Division durch 0), daher können nicht für alle Items ein Cohen's Kappa angegeben werden. In Tabelle \ref{tab:CohenKappa} sind alle Ergebnisse zusammengefasst.


\github{http://git.io/mk9z-Q}

\begin{table}[htbp]
  \centering
\begin{tabular}{@{}lllllllll@{}}
\toprule   &  \multicolumn{2}{c}{201} &&  \multicolumn{2}{c}{301}  && \multicolumn{2}{c}{301}\\
 \cmidrule{2-3}  \cmidrule{5-6} \cmidrule{8-9}
Item  & $p_0$ & $\kappa$ &&  $p_0$ & $\kappa$ &&  $p_0$ & $\kappa$\\
\midrule
 1.1 & 1 & 1  && 0.91 & 0.74 && 0.91 & 0.79 \\ 
 1.2 & 0.91 & 0.81  && 1 & /  && 1 & 1 \\ 
 2.1 & 0.81 & 0.67  && 0.81 & 0.74  && 1 & 1\\ 
 3.1 & 1 & 1  && 0.91 & 0.81  && 1 & 1\\ 
 3.2 & 1 & /  && 1 & 1  && 0.91 & 0.82\\ 
 4.1 & 0.91 & 0.79 && 0.81 & 0.65  && 0.91 & 0.81 \\ 
 4.2 & 0.91 & 0.62 && 0.91 & 0.79  && 0.91 & 0.74 \\ 
 4.3 & 1 & /  && 1 & /  && 1 & / \\ 
 4.4 & 1 & /  && 1 & /  && 1 & / \\ 
 5.1 & 1 & /  && 1 & /  && 1 & / \\ 
 5.2 & 0.91 & /  && 1 & 1  && 0.91 & 0.78 \\ 

\bottomrule

\end{tabular} 
  \caption{Übereinstimmung der Kodierungen für die einzelnen Items ($p_0$) und Cohens Kappa $\kappa$. Für die drei Tests 201 (Chemie-Temperatur), 301 (Physik Kraft) und 305 (Physik Temperatur)}
  \label{tab:CohenKappa}
\end{table}

\subsection{Qualitätsstandards}
Aus den elf Items wurden fünf Qualitätsstandards entwickelt \citep{Hild2014a}. Es gibt bedingte und unbedingte Qualitätsstandards. Bei den bedingten Qualitätsstandards ist für das erreichen notwendig, das sowohl die Bedingung erfüllt ist, als auch dass der vorgängige Qualitätsstandard erfüllt ist. Die unbedingten Qualitätsstandards werden in dieser Arbeit mit Q1 bis Q5 bezeichnet. Die bedingten Qualitätsstandards werden mit QS1 bis QS5 bezeichnet.
\subsubsection*{Qualitätsstandard 1}
Im Qualitätsstandard 1 geht es um das korrekte und präzise messen. Dieser Qualitätsstan\-dard wird nur erreicht wenn Item 1.1 (Richtige Tendenz des Resultates) und Item 1.2 (Ist das Resultat vollständig und korrekt) zusammen mindestens 1 ergeben.


\subsubsection*{Qualitätsstandard 2}
Bei Qualitätsstandard 2 wird die Dokumentation der Messung bewertet . Dieser Qualitätsstandard wird nur erreicht wenn Item 1.2 (Werden alle Messungen und Messergebnisse vollständig dargestellt) mindestens den Wert von 2 erreicht hat. 

\subsubsection*{Qualitätsstandard 3}
Im dritten Qualitätsstandard wird das Begründen des richtigen Messinstrumentes bewertet. Dieser Standard wird nur erreicht wenn Item 3.1 (Ist das Korrekte Messinstrument gewählt worden) und Item 3.2 (Wird die Wahl des Messinstrumentes korrekt begründet) zusammen zwei ergeben.

\subsubsection*{Qualitätsstandard 4}
Qualitätsstandard 4 beurteilt die Messwiederholung. Es wird aus Item 4.1 (mehrmaliges Messen), 4.2 (identische Messung), 4.3 (wurde Mittelwert gebildet) und 4.4 (korrekter Mittelwert) gebildet. Diese Level wird erreicht wenn die Items addiert mindestens zwei ergeben.

\subsubsection*{Qualitätsstandard 5}
Der letzte Qualitätsstandard 5 zeigt auf, inwiefern die Schülerinnen und Schüler Fehlerquellen der Messung begründen können. Dieser Standard besteht aus Item 5.1 (Fehlerkategorien nennen) und 5.2 (Verbesserungsvorschläge) welche zusammen mehr als eins ergeben müssen.

\subsubsection{Erreichte Qualitätsstandards}

In Tabelle \ref{tab:QS} wird ein Überblick über die erreichten Qualitätsstandards aller Schülerinnen und Schüler gegeben. Zusätzlich werden auch die bedingten Qualitätsstandards angeben, welche nur erreicht werden können, wenn der vorhergehende Qualitätsstandard erreicht wurde.


\begin{table}[!htbp]
  \centering
\begin{tabular}{@{}lllllllllll@{}}
\toprule
 Test & $p_{Q1}$ & $p_{QS1}$ & $p_{Q2}$ & $p_{QS2}$& $p_{Q3}$& $p_{QS3}$& $p_{Q4}$& $p_{QS4}$& $p_{Q5}$& $p_{QS5}$\\ 
\midrule
 201 &   0.51 & 0.51& 0.34 & 0.27 & 0.05 & 0.04 & 0.08 & 0.03 & 0.16 & 0.03 \\ 
 301 &   0.62 & 0.62& 0.31 & 0.31 & 0.09 & 0.04 & 0.09 & 0.01 & 0.39 & 0.01\\ 
 305 &   0.72 & 0.72& 0.30 & 0.29 & 0.35 & 0.14 & 0.11 & 0.01 & 0.50 & 0.01\\ 
\bottomrule
\end{tabular} 
  \caption{Zusammenfassung der erreichten Qualitätsstandards, wobei $p_{Q1} - p_{Q5}$ den unbedingten Qualitätsstandards entsprechen. Die bedingten Qualitätsstandards werden mit $p_{QS1} - p_{QS5}$ bezeichnet.}
  \label{tab:QS}
\end{table}




\subsection{Niveau}

Basierend auf den Qualitätsstandards wurden zwei Niveaus gebildet, welche das erreichte Niveau der Schülerinnen und Schüler bei der Kompetenz des skalenbasierten Messens bezeichnen. Die Niveaus können einen Wert zwischen 0 und 5 annehmen. Eine Übersicht über die erreichten Niveaus wird in Tabelle \ref{tab:Niveau} gegeben.
\github{http://git.io/bjn9qg}

\begin{table}[htbp]
  \centering
\begin{tabular}{@{}llllllllllllll@{}}
\toprule
 &  \multicolumn{6}{c}{unbedingtes Niveau} &&  \multicolumn{6}{c}{bedingtes Niveau}\\ 
 \cmidrule{2-7} \cmidrule{9-14}
 Test & 0 & 1 & 2 & 3 & 4 & 5 && 0 & 1 & 2 & 3 & 4 & 5\\ 
\midrule
 201 &   0.36 & 0.24 & 0.22 & 0.13 & 0.03 & 0.03 && 0.40 & 0.24 & 0.32  & 0.01 & 0 & 0.03   \\ 
 301 &   0.31 & 0.21 & 0.29 & 0.14 & 0.03 & 0.03  && 0.42 & 0.28 & 0.26 & 0.01 & 0 & 0.03  \\ 
 305 &   0.13 & 0.19 & 0.24 & 0.31 & 0.11 & 0.03  && 0.22 & 0.43 & 0.18 & 0.13 & 0.01 & 0.03 \\ 
\bottomrule
\end{tabular} 
  \caption{Prozedural erreichte Niveaus aller Schülerinnen und Schüler. Beim bedingten Niveau ist es jeweils erforderlich, dass alle vorhergehenden Qualitätsstandards auch erreicht worden sind.}
  \label{tab:Niveau}
\end{table}

\subsubsection{Unbedingtes Niveau}
Dieses Niveau ist der Summenscore der einzelnen unbedingten Qualitätsstandards. In der Arbeit wird dieses Level mit \textit{uLev} abgekürzt.

\subsubsection{Bedingtes Niveau}

Dieses Niveau ist der Summenscore der bedingten Qualitätsstandards. Dieses Niveau wird mit \textit{kLev} abgekürzt.




\section{Fragebogen}

Im standardisierten Teil des Fragebogens wurden Fragen zum absoluten Selbstkonzept nach SESSKO gestellt \citep{Schone2002}. Die verwendeten Fragen sind in Tabelle \ref{tab:SESSKO} aufgeführt. 



\begin{table}[htbp]
  \centering
\begin{tabular}{@{}p{3cm}p{9cm}p{1cm}@{}}
\toprule Skala & Frage & $\alpha_d$  \\ 
\midrule SESSKO 18(a) & Ich bin für die Schule sehr begabt. &  0.71  \\ 
 SESSKO 19(a) & Neues zu lernen fällt mir schwer.  &  0.76 \\ 
 SESSKO 20(a) & Ich bin sehr intelligent. &  0.71  \\ 
 SESSKO 21(a) & Ich kann in der Schule viel. &  0.72   \\ 
 SESSKO 22(a) & In der Schule fallen mir viele Aufgaben schwer.  & 0.74   \\ 
\bottomrule 
\end{tabular} 
  \caption{Fragen von SESSKO zur Skala "`Schulisches Selbstkonzept - absolut"'  \citep{Schone2002}. $\alpha_d$ bezeichnete das standardisierte Cronbach Alpha wenn dieses Item weggelassen würde.}
  \label{tab:SESSKO}
\end{table}

Zusätzlich wurden nach \citet{Dierks2014} Fragen zum Selbstkonzept zu Schulversuchen entwickelt und angepasst. Die entwickelten Fragen sind in Tabelle \ref{tab:NatSK} aufgezeigt.

\begin{table}[htbp]
  \centering
\begin{tabular}{@{}p{2cm}p{10cm}p{1cm}@{}}
\toprule Kürzel & Frage & $\alpha_d$  \\ 
\midrule NatSK1 & Schulversuche liegen mir nicht besonders. &  0.65  \\ 
 NatSK2 & Schulversuche würde ich viel lieber machen, wenn sie nicht so schwer wären.  &  0.69 \\ 
 NatSK3 & Schulversuche fallen mir schwerer als vielen meiner Mitschüler/innen. &  0.65  \\ 
 NatSK4 & Bei manchen Schulversuche weiss ich gleich: "`Das verstehe ich nie."' &  0.65   \\ 
 NatSK5 & Für Schulversuche habe ich einfach keine Begabung.   & 0.63   \\ 
 NatSK6 & Mit den Aufgaben bei Schulversuche komme ich besser zurecht als viele meiner Mitschüler/innen  & 0.67   \\ 
 NatSK7 & Ich denke, ich bin für Schulversuche begabter als viele meiner Mitschüler/innen.  & 0.66   \\ 
\bottomrule 
\end{tabular} 
  \caption{Fragen zum Sebstkonzept bei Schulversuchen abgewandelt nach \citet{Dierks2014}. $\alpha_d$ bezeichnete das standardisierte Cronbach Alpha wenn dieses Item weggelassen würde.}
  \label{tab:NatSK}
\end{table}

\label{txt:Cronbach}
Es wurde die innere Konsistenz beider Skala überprüft. Für die innere Konsistenz wurde Cronbach $\alpha$ verwendet, da dies nach \citet{Eisinga2013} eher zu einer Unterschätzung der innere Konsistenz führt. Bei den der Skala "`Schulisches Selbstkonzept - absolut"' wurde ein standardisiertes Cronbach $\alpha$ von 0.77 erreicht. Die Anzahl vollständig ausgefüllter Fragebögen betrug dabei 69. Alle unvollständigen Items wurden vor der Analyse entfernt. Bei der Skala zum Selbstkonzept bei Schulversuchen wurde ein standardisiertes Cronbach $\alpha$ von 0.69 erreicht. Insgesamt konnten dabei 64 vollständige Fragebögen ausgefüllt werden. 
\github{http://git.io/WyJH6Q}


\section{Unterschiede zwischen den Klassen}

Um festzustellen, ob alle Datensätze der einzelnen Klassen kombiniert werden dürfen wurden zuerst alle Klassen einzeln gegeneinander auf folgende Nullhypothese überprüft: 
\begin{quote}
Besteht \underline{kein} Unterschied in den Qualitätsstandards zwischen den einzelnen Klassen?
\end{quote}
Es wurden dabei die Qualitätsstandards verglichen, da diese im Vergleich zu den Items ein geringeres Rauschen aufweisen, ohne jedoch gross an Informationsgehalt eingebüsst zu haben.


Aufgrund der geringen Anzahl an Beobachtungen für einzelne Qualitätsstandards wurde der exakter Test nach Fisher verwendet. Es wurden Kontingenztafeln für jeden Qualitätsstandard (Q1 bis Q5 und QS1 bis QS5) erstellt und in jeder Tafel die beiden Levels (0 und 1) gegenüber den Klassen verglichen.


\begin{table}[htbp]
  \centering
\begin{tabular}{@{}llllllllllll@{}}
\toprule
 Klasse & Q1 & Q2 & Q3 & Q4 & Q5 && QS1 & QS2 & QS3 & QS4 & QS5 \\ 
\midrule
 1 vs. 2 &   0.68 & 1.00 & 1.00 & 0.60 & 1.00 && 0.51 & 0.59 & 1.00 & 1.00 & 1.00   \\ 
 1 vs. 3 &   1.00 & 0.72 & 1.00 & 1.00 & 1.00 && 1.00 & 0.72 & 1.00 & 1.00 & 1.00   \\
 1 vs. 4 &   0.43 & 0.72 & 0.22 & 0.32 & 0.65 && 0.42 & 0.72 & 0.48 & 1.00 & 1.00   \\
 2 vs. 3 &   0.68 & 0.72 & 1.00 & 0.22 & 1.00 && 0.68 & 1.00 & 1.00 & 1.00 & 1.00   \\
 2 vs. 4 &   1.00 & 0.72 & 0.60 & 1.00 & 1.00 && 1.00 & 1.00 & 1.00 & 1.00 & 1.00   \\
 3 vs. 4 &   0.43 & 1.00 & 0.22 & 0.10 & 0.65 && 0.43 & 1.00 & 0.48 & 1.00 & 1.00   \\
\bottomrule
\end{tabular} 
  \caption{p-Werte für den exakten Test nach Fisher für die Vergleiche der einzelnen Klassen untereinander auf allen Qualitätsstandards. Kein p-Wert in dieser Tabelle liegt unter 0.05. }
  \label{tab:KlassenVergleiche}
\end{table}


Die Resultate des exakten Tests nach Fisher befindet sich in Tabelle \ref{tab:KlassenVergleiche}. Bei keinem der 60 Tests konnte die Nullhypothese abgelehnt werden ($p < 0.05$). Daher gibt es keinen signifikanten Unterschied zwischen den erreichten Qualitätsstandards in den einzelnen Klassen.  
\github{http://git.io/0DOelQ}

\section{Korrelation der Niveaus des skalenbasierten Messens}

In einem nächsten Schritt wurde untersucht inwiefern die Niveau-Stufen (uLev und cLev) zwischen den einzelnen Tests korrelieren. Dazu wurde als Rangkorrelationskoeffizient Spearmans $\rho$ berechnet. Der Vorteil dieser Methode ist, dass keine Annahmen über die Zugrundliegeenden Daten gemacht werden muss. Des Weiteren bietet diese Methode den Vorteil, dass sie gegenüber Ausreissern robust ist \citep{Kowalski1972}. 

Da die Korrelation alleine keinen Aufschluss darüber gibt, ob diese Korrelation signifikant ist, wurde die Korrelation zusätzlich auf Signifikanz getestet. Wichtig bei dieser Analyse ist, dass die Korrelation keine Aussage über die Kausalität zulässt.


Die Ergebnisse wurden grafisch als Streudiagramme dargestellt (siehe \ref{fig:corLev}). In die Streudiagramme wurde die Gerade der linearen Regression eingetragen mit dem zugehörigen 95\% Vertrauensintervall. Zusätzlich wurde die noch Spearmans $\rho$ und der p-Wert des Signifikanztests angegeben, diese Werte sind auch in Tabelle \ref{tab:CorNiveau} zusammengefasst.



\begin{table}[htbp]
  \centering
\begin{tabular}{@{}lllllll@{}}
\toprule
 &&  \multicolumn{2}{c}{uLev} &&  \multicolumn{2}{c}{kLev}\\ 
 \cmidrule{3-4}  \cmidrule{6-7}
 Test && p-Wert & $\rho$ && p-Wert & $\rho$  \\ 
\midrule
 201 vs. 301 &&   0.02 & 0.26 && 0.20 & 0.08    \\ 
 201 vs. 305 &&   0.44 & 1e-4 && 0.33 & 4e-3      \\
 301 vs. 305 &&   0.36 & 2e-3 && 0.01 & 0.89    \\
\bottomrule
\end{tabular} 
  \caption{Spearmans $\rho$ und p-Werte für die Korrelation zwischen den unbedingten Niveaus (uLev) und den bedingten Niveaus (kLev) zwischen den einzelnen Tests.  }
  \label{tab:CorNiveau}
\end{table}
 
\begin{figure}[htbp]
\centering
\begin{subfigure}{0.49\textwidth}
  \includegraphics[width=1.0\linewidth]{graphics/cor201301u.png}
  \caption{Korrelation der unbedingten Niveau-Stufen zwischen Test 301 und 201.}
  \label{fig:cor201301k}
\end{subfigure}
\begin{subfigure}{0.49\textwidth}
  \includegraphics[width=1.0\linewidth]{graphics/cor201301k.png}
  \caption{Korrelation der bedingten Niveau-Stufen zwischen Test 301 und 201.}
  \label{fig:cor201301u}
\end{subfigure}
\end{figure}
\begin{figure}[htbp]
\ContinuedFloat % continue from previous page
\centering
\begin{subfigure}{0.49\textwidth}
  \includegraphics[width=1.0\linewidth]{graphics/cor201305u.png}
  \caption{Korrelation der unbedingten Niveau-Stufen zwischen Test 305 und 201.}
  \label{fig:cor201305k}
\end{subfigure}
\begin{subfigure}{0.49\textwidth}
  \includegraphics[width=1.0\linewidth]{graphics/cor201305k.png}
  \caption{Korrelation der bedingten Niveau-Stufen zwischen Test 305 und 201.}
  \label{fig:cor201305u}
\end{subfigure}
\end{figure}
\begin{figure}[htbp]
\ContinuedFloat % continue from previous page
\centering
\begin{subfigure}{0.49\textwidth}
  \includegraphics[width=1.0\linewidth]{graphics/cor301305u.png}
  \caption{Korrelation der unbedingten Niveau-Stufen zwischen Test 305 und 301.}
  \label{fig:cor301305k}
\end{subfigure}
\begin{subfigure}{0.49\textwidth}
  \includegraphics[width=1.0\linewidth]{graphics/cor301305k.png}
  \caption{Korrelation der bedingten Niveau-Stufen zwischen Test 305 und 301.}
  \label{fig:cor301305u}
\end{subfigure}

\caption{Korrelation zwischen den Niveau-Stufen der einzelnen Tests. Der Durchmesser der Punkte ist ein Mass für die Anzahl an Datenpunkten, welche an dieser Position liegen. Die blaue Gerade ist die lineare Regression der zugrunde liegenden Daten, der dunkel graue Bereich stellt das Vertrauensintervall (95\%) der linearen Regression dar. Zusätzlich sind noch Spearmans $rho$ und der p-Wert des Signifikanztests angegeben.}
\label{fig:corLev}
\end{figure}

\github{http://git.io/FnbD}

\clearpage
\section{Rasch-Analyse}

Als Probabilistische Test-Methode wurde das Rasch Modell verwendet. Der Grund für diese Methodik war, dass es sich bei der Kompetenz des skalenbasierten Messens um ein latentes Merkmal handelt. In anderen Worten die Kompetenz des skalenbaiserten Messens ist nicht direkt beobachtbar.

Es wurde zuerst folgendes Rasch Modell verwendet.

\begin{eqnarray}
P(U_{ij}=u_{ij}|\theta_i,\beta_j) = \frac{e^{u_{ij}(\theta_i-\beta_j)}}{1+e^{\theta_i-\beta_j}}
\end{eqnarray}

Wobei i=1,…,n die Zählvariable für eine Person ist und j=1,…,m die Zählvariable für eine Aufgabe darstellt. Die Variable $u_{ij} \in \{0,1\}$ die dichotome Antwort einer Person auf eine Frage ist. Die Variable $\beta_j$ beschreibt den Schwierigkeitsgrad einer Aufgabe und $\theta_j$ die latente Fähigkeit einer Person.

Bei der Item-Response-Theorie (Probabilistische Test-Methoden) wird angenommen , dass das Ergebnis einer Person nicht deterministisch ist, sondern zufällig sein kann. Daher soll mit dem Rasch Modell die Lösungswahrscheinlichkeit jeder Aufgabe $U_{ij}$ berechnet werden. Diese Lösungswahrscheinlichkeit hängt sowohl von der Fähigkeit der Person $\theta_j$ als auch von der Schwierigkeit der Aufgabe $\beta_i$ ab. Diese Lösungswahrscheinlichkeiten werden basierend auf den Testergebnissen $u_{ij}$ geschätzt.

\subsection{Parameterschätzung}
Für die Parameterschätzung gibt es verschieden Ansätze. Da die beste Methode von den Daten abhängig ist wurde in einem ersten Schritt das Rasch-Modell sowohl mit der bedingte Maximum-Likelihood-Schätzung, als auch mit der marginal Maximum-Likelihood-Schätzung getestet und die Resultate wurden verglichen. 

Bei der bedingten Maximum-Likelihood-Schätzung wird ein zweistufiges Vorgehen gewählt. Zuerst werden die Aufgaben-Parameter geschätzt ohne die Personen Parameter zu beachten. Erst in einem zweiten Schritt werden die Personen-Parameter geschätzt. Ein Problem dieser Methodik ist, dass Personenfertigkeiten von Personen, welche keine oder alle Aufgaben gelöst haben nicht geschätzt werden können \citep{Mair2007}.

In der marginalen Maximum-Likelihood-Schätzung wird angenommen, dass für die Personenfähigkeiten in der Stichprobe eine Normalverteilung vorliegt. Diese Annahme ist insbesondere dann problematisch, wenn nur eine Stichprobe der Gesamtbevölkerung verwendet wird \citep{Rizopoulos2006}.

Da beide Schätzungen für den vorliegenden Datensatz problematisch sein könnten, wurde das Rasch-Modell mit beiden Ansätzen durchgeführt und die Resultate verglichen. Das Ziel war dabei, den besseren Ansatz für den vorliegenden Datensatz zu finden, um mit diesem Ansatz die weiteren Analysen durchzuführen. Als Datensatz für diesen Vergleich wurden die 15 unbedingten Qualitätsstandards verwendet. Die Resultate sind in Abbildung \ref{fig:RaschVergleich} ersichtlich. Es gibt für diesen Datensatz keinerlei Unterschied in der Schätzung der Schwierigkeitsgrad der einzelnen Qualitätsstandards. 

Bei der Schätzung der Personen-Parametern konnte die bedingten Maximum-Likelihood-Schätzung alle 72 Personen-Fähigkeiten ohne Extrapolationen berechnen. Die marginalen Maximum-Likelihood-Schätzung konnte jedoch nur die Personen-Fähigkeiten von 64 Personen berechnen. Daher wird in der weiteren Arbeit für alle Rasch Modelle jeweils der bedingten Maximum-Likelihood-Schätzer verwendet.


\begin{figure}[htbp]

\centering
\includegraphics[width=0.6\linewidth]{graphics/RaschVergleich.png}
\caption{Vergleich des Rasch Modells mit der bedingten Maximum-Likelihood-Schätzung und der marginalen Maximum-Likelihood-Schätzung. Da alle Punkte auf einer Geraden liegen, gibt es keinen Unterschied zwischen den unterschiedlichen Schätzmethoden für den vorliegenden Datensatz der 15 unbedingten Qualitätsstandards.  }
\label{fig:RaschVergleich}
\end{figure}


Es gibt noch weitere Parameter Schätzer wie den Bayesianischen Ansatz, welcher Markov-Chain-Monte-Carlo Methoden verwendet. Dieser trifft jedoch auch Annahmen über die Verteilung der Personen Parameter \citep[siehe Kapitel 3]{Fischer1995}. Die Annahmen decken sich daher mit dem marginalen Maximum-Likelihood Schätzer. 


\github{http://git.io/FRxZ}


\subsection{Modellkontrolle des Rasch-Modells}

Um das Rasch Modell zu Validieren wurde das Modell mit Hilfe des Andersens Likelihood-Quotienten Test validiert. Für alle 15 Qualitätsstufen führte dies zu Problemen und der Test konnte nicht durchgeführt werden. Nachdem die Qualitätsstufen vier und fünf entfernt wurden, konnte das reduzierte Modell validiert werden. Als Splitkriterium wurde der Mittelwert der Personen-Randsummen verwendet. 

Der p-Wert des Andersens Likelihood-Quotienten Test beträgt $p=0.14$. Daher liegt jetzt keine signifikante Modellverletzung vor, die Aufgaben Parameter unterscheiden sich nicht signifikant für Personen mit niedrigen und hohen Randsummen. In der Grafik \ref{fig:RaschKontrolle} sind die Resultate des Tests grafisch dargestellt. Es ist ersichtlich, dass keine Aufgabe das Modell verletzt, da die 95\%-Konfidenz-Regionen alle die Diagonale berühren.


\begin{figure}[htbp]

\centering
\includegraphics[width=0.8\linewidth]{graphics/GOFQ.png}
\caption{Modellkontrolle des Rasch-Modells: kein Qualitätsstandard hat eine signifikante Abweichung von der Diagonalen, daher gibt es keine signifikanten Unterschiede für Personen mit niedrigen und hohen Randsummen in den Qualitätsstandard. }
\label{fig:RaschKontrolle}
\end{figure}

Zusätzlich wurden die Qualitätsstandards mit dem Wald-Test überprüft. Damit können Qualitätsstandard, welche einen signifikanten Unterschied habe, identifiziert werden. In Tabelle \ref{tab:WaldTest} befinden sich die p-Werte des Wald-Test für die einzelnen Qualitätsstandards.

\begin{table}[htbp]
  \centering
\begin{tabular}{@{}lllllllllll@{}}
\toprule
 \multicolumn{3}{c}{Test 201} &&  \multicolumn{3}{c}{Test 301}&&  \multicolumn{3}{c}{Test 305}\\ 
    \cmidrule{1-3}\cmidrule{5-7}\cmidrule{9-11}
 Q1 & Q2 & Q3 && Q1 & Q2 & Q3 && Q1 & Q2 & Q3  \\ 
\midrule
  0.44 & 0.08 & 0.24 && 0.11 & 0.33 & 0.56 && 0.38 & 0.14 & 0.61   \\ 

\bottomrule
\end{tabular} 
  \caption{p-Werte des Wald-Tests für die Qualitätsstandards, mit dem Mittelwert der Personen-Randsummen als Splitkriterium. Keine dieser p-Werte liegt unter halb von $0.05$ daher gibt es keine signifikanten Unterschiede in den Qualitätsstandards }
  \label{tab:WaldTest}
\end{table}

\github{http://git.io/FE3m}

\subsection{Unterschied in den Qualitätsstandards}

Nachdem das Modell kontrolliert wurde soll nun überprüft werden ob es einen Unterschied in den Qualitätsstandards zwischen den einzelnen Test gibt.


  
 \begin{figure}[htp]
 \centering
 \begin{subfigure}{0.49\textwidth}
   \includegraphics[width=1.0\linewidth]{graphics/ICCQ1.png}
   \caption{ICC Plot für Qualitätsstandard 1}
   \label{fig:ICCQ1}
 \end{subfigure}
 \begin{subfigure}{0.49\textwidth}
   \includegraphics[width=1.0\linewidth]{graphics/ICCQ2.png}
   \caption{ICC Plot für Qualitätsstandard 2}
   \label{fig:ICCQ2}
 \end{subfigure}
 \end{figure}
 \begin{figure}[htbp]
 \ContinuedFloat % continue from previous page
 \centering
 \begin{subfigure}{0.49\textwidth}
   \includegraphics[width=1.0\linewidth]{graphics/ICCQ3.png}
   \caption{ICC Plot für Qualitätsstandard 3}
   \label{fig:ICCQ3}
 \end{subfigure}
 \begin{subfigure}{0.49\textwidth}
   \includegraphics[width=1.0\linewidth]{graphics/ICCQ123.png}
   \caption{ICC Plot für Qualitätsstandard 1,2 und 3}
   \label{fig:ICCQ123}
 \end{subfigure}
 
 \caption{Aufgabencharakteristische Kurven für die Qualitätsstandards 1,2 und 3 für alle drei Tests.}
 \label{fig:corLevRasch}
 \end{figure}
 
 \begin{figure}[htbp]
 
 \centering
 \includegraphics[width=0.8\linewidth]{graphics/PersonItemMap.png}
 \caption{Person-Item-Map auf welcher die Verteilung der Personen basierend auf der latenten Skala ersichtlich ist und die Lage der Aufgaben-Parameter auf der latenten Skala. Anhand dieser Darstellung kann man sehen, dass z.B. der Qualitätsstandard 1 im Test 201 und im Test 301 einen sehr ähnlichen Schwierigkeitsgrad besitzen. }
 \label{fig:PersonItemMapQ}
 \end{figure}
 
 In Tabelle \ref{tab:betaQ} finden sich die Aufgaben-Parameter $\beta_j$ der einzelnen Qualitätsstandards.
 
 \begin{table}[htbp]
   \centering
 \begin{tabular}{@{}lllllllllll@{}}
 \toprule
  \multicolumn{3}{c}{Test 201} &&  \multicolumn{3}{c}{Test 301}&&  \multicolumn{3}{c}{Test 305}\\ 
   \cmidrule{1-3}\cmidrule{5-7}\cmidrule{9-11}
  Q1 & Q2 & Q3 && Q1 & Q2 & Q3 && Q1 & Q2 & Q3  \\ 
 \midrule
   1.215 & 0.159 & -2.633 && 1.142 & -0.278 & -2.086 && 2.305 & 0.017 & 0.159   \\ 
 
 \bottomrule
 \end{tabular} 
   \caption{Aufgaben-Parameter $\beta_j$ für die einzelnen Qualitätsstandards. }
   \label{tab:betaQ}
 \end{table}
 
 Mit den so gewonnen Aufgaben-Parametern $\beta_j$ wurde nun die Korrelation zwischen den einzelnen Test berechnet. Da mit dem bisherigen Rasch Modell der Personen-Parameter $\theta_i$ über alle drei Tests identisch ist, sollten sich die Schwierigkeitsgerade der einzelnen Qualitätsstufen in den Tests sich nicht unterscheiden. Die Ergebnisse dieser Analyse sind im der Darstellung \ref{fig:corTestQ} und in Tabelle \ref{tab:corTestQ} angegeben. Wichtig dabei ist, dass der Stichproben Umfang mit 3 sehr gering ist. 
 
 
  \begin{figure}[htp]
  \centering
  \begin{subfigure}{0.32\textwidth}
    \includegraphics[width=1.0\linewidth]{graphics/GOF201301.png}
    \caption{Vergleich der Aufgaben-Parameter $\beta_j$ zwischen Test 201 und 301.}
    \label{fig:cor201301}
  \end{subfigure}
  \begin{subfigure}{0.32\textwidth}
    \includegraphics[width=1.0\linewidth]{graphics/GOF201305.png}
    \caption{Vergleich der Aufgaben-Parameter $\beta_j$ zwischen Test 201 und 305.}
    \label{fig:cor201305}
  \end{subfigure}
  \begin{subfigure}{0.32\textwidth}
    \includegraphics[width=1.0\linewidth]{graphics/GOF301305.png}
    \caption{Vergleich der Aufgaben-Parameter $\beta_j$ zwischen Test 301 und 305.}
    \label{fig:cor301305}
  \end{subfigure}
  
  \caption{Vergleich der Aufgaben Parameter zwischen den einzelnen Test. Wenn die Schwierigkeiten der Qualitätsstandards übereinstimmen würden, müssten alle Punkte auf der Winke-Halbierenden liegen.}
  \label{fig:corTestQ}
  \end{figure}
 
  \begin{table}[htbp]
    \centering
  \begin{tabular}{@{}llllllll@{}}
  \toprule
   \multicolumn{2}{c}{201 vs 301} &&  \multicolumn{2}{c}{201 vs 305}&&  \multicolumn{2}{c}{301 vs 305}\\ 
      \cmidrule{1-2} \cmidrule{4-5} \cmidrule{7-8}   
   p-Wert & ks && p-Wert & ks && p-Wert & ks  \\ 
  \midrule
    1.00 & 0.33 && 1.00 & 0.33 && 0.60 & 0.67  \\ 
  
  \bottomrule
  \end{tabular} 
    \caption{Resultate des Kolmogorow-Smirnow-Test für die Übereinstimmung der Schwierigkeiten der Qualitätsstandards. Wobei ks die Test-Statistik des Kolmogorow-Smirnow-Test ist.}
    \label{tab:corTestQ}
  \end{table}
  
  \github{http://git.io/FVZt}
  
  
\subsection{Unterschied in den latenten Personen-Fähigkeiten}

Nachdem in einem ersten Schritt die Schwierigkeit der Qualitätsstandards untersucht wurde und festgestellt wurde, dass keine signifikante Unterschiede in den Schwierigkeitsgeraden zwischen den einzelnen Tests existieren, wurde nun ein neues Rasch-Modell entwickelt.

Es werden jetzt drei Rasch Modelle gebildet, bei denen jeder Test und dessen Qualitätsstandards 1-3 in einem Modell kombiniert wurden. Aus den drei Modellen wurden die Personen-Fähigkeiten berechnet und dann mit dem Kolmogorow-Smirnow-Test auf den Goodness of fit überprüft. Dabei wurden Personen Parameter, welche Aufgrund des bedingten Maximum-Likelihood Schätzers nicht berechnet werden konnten aus den Daten heraus gefiltert. Wichtig hierbei ist jedoch, dass diese drei Rasch Modelle aufgrund der Probleme mit dem Parameter Schätzer, nicht evaluiert werden konnten, da die Datensätze zu gering waren. Um diesen Vergleich sinnvoll durchzuführen bräuchte es einen neuen besseren Schätzer. Der marginal Maximum-Likelihood Schätzer konnte deutlich weniger Personen Parameter schätzen, als der bedinge Maximum-Likelihood Schätzer. 

Die Ergebnisse der Test befinden sich in den Darstellungen \ref{fig:GOFP} und die wichtigesten Test Parameter in Tabelle \ref{tab:GOFP}.
   \begin{table}[htbp]
     \centering
   \begin{tabular}{@{}lllllllllll@{}}
   \toprule
    \multicolumn{3}{c}{201 vs 301} &&  \multicolumn{3}{c}{201 vs 305}&&  \multicolumn{3}{c}{301 vs 305}\\ 
       \cmidrule{1-3}\cmidrule{5-7}\cmidrule{9-11}
    p-Wert & ks & n && p-Wert & ks & n && p-Wert & ks & n \\ 
   \midrule
     2e-3 & 0.45 & 33&& 1e-6 & 0.62 & 37 && 1e-3 & 0.60 & 20  \\ 
   
   \bottomrule
   \end{tabular} 
     \caption{Resultate des Kolmogorow-Smirnow-Test für die Übereinstimmung der Personen-Parameter zwischen den drei Tests. Wobei ks die Test-Statistik des Kolmogorow-Smirnow-Test ist. Mit $n$ wird die Anzahl an Personenparametern angegeben, welche für den Test verwendet werden konnten. }
     \label{tab:GOFP}
   \end{table}
   

   
 \begin{figure}[htp]
 \centering
 \begin{subfigure}{0.49\textwidth}
   \includegraphics[width=1.0\linewidth]{graphics/GOF201301Pers.png}
   \caption{Vergleich der Personen-Parameter $\theta_i$ zwischen Test 201 und 301.}
   \label{fig:GOF201301P}
 \end{subfigure}
 \begin{subfigure}{0.49\textwidth}
   \includegraphics[width=1.0\linewidth]{graphics/GOF201305Pers.png}
   \caption{Vergleich der Personen-Parameter $\theta_i$ zwischen Test 201 und 305.}
   \label{fig:GOF201305P}
 \end{subfigure}

  \begin{subfigure}{0.5\textwidth}
    \includegraphics[width=1.0\linewidth]{graphics/GOF301305Pers.png}
    \caption{Vergleich der Personen-Parameter $\theta_i$ zwischen Test 301 und 305.}
    \label{fig:GOF301305P}
  \end{subfigure}
 
 \caption{Vergleich der Personenparameter für die drei Tests. Zusätzlich sind der P-Wert des Kolmogorow-Smirnow-Test und die Test-Statistik $ks$ angegeben.}
 \label{fig:GOFP}
 \end{figure}
 
\github{http://git.io/FVjL}
\clearpage 
\subsection{Zusammenhang Rasch Modell und Fragebogen}

Hierfür wurde wieder das erste Rasch Modell verwendet, bei dem die Qualitätsstandards 1 bis 3 als Items verwendet wurden und pro Person nur eine Personen-Fähigkeit geschätzt wurde.
Die so geschätzten Personen-Fähigkeiten wurden mit den Ergebnissen des Fragebogens korreliert. Die Resultate befinden sich in den Darstellungen  und die Testergebnisse nochmals zusammengefasst in Tabelle.

\begin{table}[htbp]
  \centering
\begin{tabular}{@{}lllllllllll@{}}
\toprule
   \multicolumn{2}{l}{Note Mathe} &&  \multicolumn{2}{l}{Note NatW.}&&  \multicolumn{2}{l}{SESSKO}&&  \multicolumn{2}{l}{Selbskonzept Schulversuche}\\ 
      \cmidrule{1-2}\cmidrule{4-5}\cmidrule{7-8}\cmidrule{10-11}
   p-Wert & $\rho$ && p-Wert & $\rho$  && p-Wert & $\rho$&& p-Wert & $\rho$\\ 
\midrule
   0.16 & 0.17 && 0.95 & 0.0 && 0.46 & 0.09 && 0.04 & 0.23    \\ 

\bottomrule
\end{tabular} 
  \caption{Spearmans $\rho$ und p-Werte für die Korrelation zwischen der Personen-Fähigkeit $\theta$ und verschiedenen Skalen.  }
  \label{tab:CorPersonRasch}
\end{table}

 \begin{figure}[htp]
 \centering
 \begin{subfigure}{0.49\textwidth}
   \includegraphics[width=1.0\linewidth]{graphics/corPersonenMathe.png}
   \caption{Korrelation der Personen-Fähigkeit $\theta$ mit der Note in Mathe.}
   \label{fig:corPersonenMathe}
 \end{subfigure}
 \begin{subfigure}{0.49\textwidth}
   \includegraphics[width=1.0\linewidth]{graphics/corPersonenNatw.png}
   \caption{Korrelation der Personen-Fähigkeit $\theta$ mit der Note in den Naturwissenschaften.}
   \label{fig:corPersonenNatW}
 \end{subfigure}
  \end{figure}
  \begin{figure}[htbp]
  \ContinuedFloat % continue from previous page
  \centering
  \begin{subfigure}{0.49\textwidth}
   \includegraphics[width=1.0\linewidth]{graphics/corPersonenSESSKO.png}
   \caption{Korrelation der Personen-Fähigkeit $\theta$ mit dem absoluten Schulischem Selbstkonzept nach SESSKO \citep{Schone2002}.}
   \label{fig:corPersonenSESSKO}
  \end{subfigure}
  \begin{subfigure}{0.49\textwidth}
    \includegraphics[width=1.0\linewidth]{graphics/corPersonenSelbskonzept.png}
    \caption{Korrelation der Personen-Fähigkeit $\theta$ mit dem Selbstkonzept bei Schulversuchen.}
    \label{fig:corPersonenSelbskonzept}
  \end{subfigure}
 \caption{Korrelation zwischen der Personen-Fähigkeit $\theta$ und verschiedenen per Fragebogen erhobenen Daten. Der Durchmesser der Punkte ist ein Mass für die Anzahl an Datenpunkten, welche an dieser Position liegen. Die blaue Gerade ist die lineare Regression der zugrunde liegenden Daten, der dunkel graue Bereich stellt das Vertrauensintervall (95\%) der linearen Regression dar. Zusätzlich sind noch Spearmans $rho$ und der p-Wert des Signifikanztests angegeben.}
 \label{fig:corPersonen}
 \end{figure}
 \github{http://git.io/FwCx}
 
 
 \section{Videoanalyse}
 
 Insgesamt waren vier Stunden Videomaterial angefallen. Es wurde wie bereits erwähnt nur in einer Halbklasse eine Videoaufnahme durchgeführt. Aufgrund der Position der Videokamera konnten nur die Aktionen von je zwei Schülern und Schülerinnen analysiert werden. So konnten von 8 Schülerinnen und Schülern die Aktionen per Video analysiert werden.
 
 \subsection{Qualitätsstandards}
 
 Es wurden die existierende Qualitätsstandards auf Überprüfbarkeit per Video analysiert. Es wurden die Qualitätsstandards 1 und 4 als analysierbar identifiziert. Für diese beiden Standards wurde jeweils eine Kodierung definiert.
 
 \subsubsection{Korrekt und präzise messen}
 
 Es wurde eine Kodierung, welche an \citet{Schreiber2012} angelehnt war verwendet. Bei der Kodierung des Merkmales korrekt und präzise messen wurde von einer Gütestufe von 3 ausgegangen. Wenn ein Schüler oder Schülerin nicht korrekt abgemessen hat (z.B. schräg abgelesen), wurde Gütestufe 2 kodiert. Wenn die Schülerin oder der Schüler bei den einzelnen Messungen unterschiedlich gemessen hat, wurde die Gütestufe 1 vergeben.
 
 \subsubsection{Messung wiederholen}
 
 Bei diesem Merkmal wurde von einer Gütestufe von 1 ausgegangen. Wenn der Schüler oder die Schülerin die Messung wiederholt hat, wurde die Gütestufe 2 erreicht. Als Messwiederholung wurde eine Messung in einem neuen Experiment definiert. Es reichte also nicht, mehrmals die den Messwert abzulesen um diese Gütestufe zu erreichen, sondern es musste das Experiment erneut durchgeführt werden. Gütestufe 3 wurde erreicht, wenn das Experiment identisch durchgeführt wurde.
 
 Die Resultate der Kodierungen befinden sich in Tabelle \ref{tab:VideoMerkmale}.
 
 
 
 \begin{table}[htbp]
   \centering
 \begin{tabular}{@{}lllllllll@{}}
 \toprule
    &&\multicolumn{3}{l}{Messung korrekt} &&  \multicolumn{3}{l}{Messwiederholung.}\\ 
       \cmidrule{3-5}\cmidrule{7-9}
    Test && 1 & 2 & 3 && 1 & 2 & 3 \\ 
 \midrule
    201 && 0.25 & 0.63 & 0.13 && 0.63 & 0.38 & 0.00   \\ 
    301 && 0.13 & 0.75 & 0.13 && 0.63 & 0.38 & 0.00   \\ 
    305 && 0.25 & 0.63 & 0.13 && 0.38 & 0.38 & 0.25\\ 
 
 \bottomrule
 \end{tabular} 
   \caption{Die erreichten Gütestufen für die Merkmale Messung wiederholen und korrekt und präzise messen. Die Anzahl kodierter Personen beträgt 8.  }
   \label{tab:VideoMerkmale}
 \end{table}
 

 
 \subsection{Korrelation zwischen Video Merkmalen und Qualitätsstufen}
 
 Da die Videokodierung Merkmale basierend auf den Qualitätsstandards entwickelt hat, wurde untersucht ob zwischen den Merkmalen und den Qualitätsstandards eine Korrelation existiert. Diese Resultate befinden sich in Darstellung \ref{fig:corVideoQ} und in Tabelle \ref{tab:CorVideoQ}. In keinem der Korrelationstests wird die Signifikanz-schwelle überschritten, daher gibt es keine signifikante Korrelation zwischen den Qualitätsstandards und den Merkmalen der Videokodierung.
 
 \begin{figure}[htp]
  \centering
  \begin{subfigure}{0.32\textwidth}
    \includegraphics[width=1.0\linewidth]{graphics/corVideoQ1201.png}
    \caption{Vergleich Q1 mit Merkmal korrekte Messung für Test 201.}
    \label{fig:corVideoQ1201}
  \end{subfigure}
  \begin{subfigure}{0.32\textwidth}
    \includegraphics[width=1.0\linewidth]{graphics/corVideoQ1301.png}
    \caption{Vergleich Q1 mit Merkmal korrekte Messung für Test 301.}
    \label{fig:corVideoQ1301}
  \end{subfigure}
  \begin{subfigure}{0.32\textwidth}
      \includegraphics[width=1.0\linewidth]{graphics/corVideoQ1305.png}
    \caption{Vergleich Q1 mit Merkmal korrekte Messung für Test 305.}
    \label{fig:corVideoQ1305}
   \end{subfigure}
 
   \begin{subfigure}{0.32\textwidth}
       \includegraphics[width=1.0\linewidth]{graphics/corVideoQ4201.png}
       \caption{Vergleich Q4 mit Merkmal korrekte Messung für Test 201.}
       \label{fig:corVideoQ4201}
     \end{subfigure}
     \begin{subfigure}{0.32\textwidth}
       \includegraphics[width=1.0\linewidth]{graphics/corVideoQ4301.png}
       \caption{Vergleich Q4 mit Merkmal korrekte Messung für Test 301.}
       \label{fig:corVideoQ4301}
     \end{subfigure}
      \begin{subfigure}{0.32\textwidth}
         \includegraphics[width=1.0\linewidth]{graphics/corVideoQ4305.png}
       \caption{Vergleich Q4 mit Merkmal korrekte Messung für Test 305.}
       \label{fig:corVideoQ4305}
       \end{subfigure}
  
  \caption{Vergleich der Merkmale der Videokodierung mit den Qualitätsstandards 1 und 4. Der Durchmesser der Punkte ist ein Mass für die Anzahl an Datenpunkten, welche an dieser Position liegen. Die blaue Gerade ist die lineare Regression der zugrunde liegenden Daten, der dunkel graue Bereich stellt das Vertrauensintervall (95\%) der linearen Regression dar. Zusätzlich sind noch Spearmans $rho$ und der p-Wert des Signifikanztests angegeben.}
  \label{fig:corVideoQ}
  \end{figure}
 
 \begin{table}[htbp]
   \centering
 \begin{tabular}{@{}lllllllllll@{}}
 \toprule
    \multicolumn{2}{l}{201 Q1} &&  \multicolumn{2}{l}{201 Q4}&&  \multicolumn{2}{l}{301 Q1}&&  \multicolumn{2}{l}{301 Q4} \\ 
       \cmidrule{1-2}\cmidrule{4-5}\cmidrule{7-8}\cmidrule{10-11}
    p-Wert & $\rho$ && p-Wert & $\rho$  && p-Wert & $\rho$&& p-Wert & $\rho$\\ 
 \midrule
    0.76 & -0.13 && 1.00 & 0.0 && 1.00 & 0.0 && 1.00 & 0.0     \\ 
 
 \bottomrule
 \end{tabular} \\[0.75cm]
 \begin{tabular}{@{}lllll@{}}
  \toprule
       \multicolumn{2}{l}{305 Q1}&&  \multicolumn{2}{l}{305 Q4}\\ 
        \cmidrule{1-2}\cmidrule{4-5}
      p-Wert & $\rho$&& p-Wert & $\rho$\\ 
  \midrule
      0.53 & 0.26 && 0.87 & 0.07    \\ 
  
  \bottomrule
  \end{tabular} 
   \caption{Spearmans $\rho$ und p-Werte für die Korrelation zwischen Qualitätsstandards und den Merkmalen aus der Videokodierung..  }
   \label{tab:CorVideoQ}
 \end{table}
 
  \subsection{Messzeitpunkte und Messdauer}
  
  Zusätzlich zu den zwei Merkmalen wurde für jede Messung noch erhoben, wann die Messung begonnen hatte und wann die Messung beendet wurde. Bei der Temperaturmessung war die Definition der Messung nicht trivial. Es wurde folgende Definition für eine Messung verwendet. Für eine Temperaturmessung, muss dass Thermometer aus dem Medium entfernt werden und abgelesen werden. Ein Ablesen ohne, dass das Thermometer aus dem Medium herausgenommen wird, gilt nicht als Messung. Der Hauptgrund für diese eingeschränkte Definition ist, dass der Ablese-Vorgang nur sehr schwierig eindeutig beobachtbar ist. Daher wurde dieser mit dem Entfernen des Thermometers verknüpft, sodass die Kodierung einfacher ist. Ein Problem dabei war der Test 201, da dort die Thermometer über das Video nicht unterscheidbar waren. Daher wurden dort die Messinstrumente mit 1 und 2 kodiert. Die Resultate finden sich in Darstellung \ref{fig:Messungen}.
  
  
  
   \begin{figure}[htp]
   \centering
   \includegraphics[width=1.0\linewidth]{graphics/Messungen2.png}
   \caption{Übersicht über alle Messungen der Videografierten Schülerinnen und Schülern. In der ersten Spalte die der Identifizierungs-Code. In der Spalte 2 welcher Test wobei gilt 1 = 305, 2 = 201, 3 = 301. In Schwarz wird jeweils markiert wenn eine Messung durchgeführt wird. Die Linie in der Mitte entspricht der Halbzeit des Testes (10 min)}
   \label{fig:Messungen}
   \end{figure}
   
   \github{http://git.io/bvQS}
  
\chapter{Diskussion}


Nachdem im letzten Kapitel die Ergebnisse präsentiert wurden, soll in diesem Kapitel versucht werden mit Hilfe der Ergebnisse die Forschungsfrage zu beantworten.

\section{Kodierung}

\subsection{Items}

Da sowohl die Qualitätsstandards als auch die Niveaus auf den Items basieren, ist eine gute Kodierung derselbigen elementar für diese. Durch die Zweitkodierung der Items sollte sichergestellt werden, dass die Kodierung der Items verlässlich und wiederholbar ist. In Tabelle \ref{tab:CohenKappa} sind die Ergebnisse für die Interrater-Reliabilität aufgeführt. Bis auf wenige Ausnahmen befinden sich alle Werte oberhalb von $\kappa > 0.75$ was nach \citet[S.111]{Greve1997} sehr gut bis ausgezeichnet ist. \citet{Landis1977} bezeichnet jedoch auch die niedrigen $\kappa$-Werte bei denen $\kappa > 0.61$ ist als "`substantial strength of agreement'. 

Ein Problem bei der Kodierung der Items und der Überprüfung, war jedoch, dass viele Schülerinnen und Schüler bestimmte Items nicht erreichten. Daher konnte Cohen's $\kappa$ nicht für alle Items berechnet werden. Da die prozedurale Übereinstimmung dort jedoch sehr hoch war, kann auch bei diesen Items von einer korrekten Kodierung ausgegangen werden. Dieses Problem kann auch eine Erklärung für die sehr gute Übereinstimmung bei bestimmten Items sein. So war es meistens sehr klar, wenn ein Schüler oder eine Schülerin ein Item nicht erreicht hatten. Daher war die Kodierung meistens sehr eindeutig.

Aufgrund dieser Ergebnisse kann davon ausgegangen werden, dass die Zweitkodierung aller Schülerinnen und Schüler keine deutlich abweichende Resultate geliefert hätten und daher die Zweitkodierung von 15\% der Schülerinnen und Schüler ausreichend war um die Qualität und Reliabilität der Kodierung festzustellen.

Daher kann davon ausgegangen werden, dass die Reliabilität der Kodierung gegeben ist und die Kodierung korrekt und nachvollziehbar ist.

\subsection{Qualitätsstandards}

Ein Problem bei der Definition der Qualitätsstandards ist die Unterschiedliche Definition in der Literatur. So verwendete \citet{Gut2013a} noch eine andere Reihenfolge der Qualitätsstandards. Die in dieser Arbeit verwendete Reihenfolge der Qualitätsstandards basiert auf den Arbeiten von \citet{Metzger2013, Hild2014a}. Ein Problem dabei ist jedoch, dass die Schwellenwerte für das Erreichen der Qualitätsstandards nicht publiziert sind. Die Schwellenwerte wurde daher von internen Dokumenten von Pitt Hild übernommen.

Die erreichten Qualitätsstandards in Tabelle \ref{tab:QS} zeigen, dass insbesondere die Qualitäts-standards 3, 4 und 5 nur von einem geringen Prozentsatz der Schülerinnen und Schüler erreicht werden. Und es auch einen Unterschied in den erreichten Qualitätsstandards zwischen den einzelnen Test gibt. In dieser Arbeit wird nicht auf diese Unterschiede eingegangen. Dafür sei auf folgende Arbeit hingewiesen \citet{Sichau2015}. Diesen Unterschied in den erreichten Qualitätsstandards deckt sich jedoch mit den Ergebnissen von \citet{Metzger2013}.

\subsection{Niveaus}

Dieses schlechte Abschneiden der Klassen spiegelt sich auch in den erreichten Niveaus wieder. So sieht man in Tabelle \ref{tab:Niveau}, dass ein Grossteil der Schülerinnen und Schüler nicht über das Niveau 2 hinauskommen, sowohl beim unbedingten als auch beim bedingten Niveau. Im Vergleich zu \citet{Metzger2013} scheiden die Schülerinnen und Schüler in der 7. Klasse schlechter ab. 

Da leider der Zeitpunkt der Datenerhebung in der Arbeit von \citet{Metzger2013} nicht aufgeführt ist, ist nicht klar ob der frühe Zeitpunkt des Testes (beginn des ersten Halbjahres) einen eventuellen Einfluss auf das Abschneiden der Schülerinnen und Schüler hatte. So war dies bei allen Klassen bei denen diese Tests durchgeführt wurden, das erste Mal, dass sie in der Oberstufe experimentiert haben. Auch kannten die Schülerinnen und Schüler den Kraftmesser nicht und konnten nur durch ausprobieren herausfinden, wie dieser funktioniert. Daher ist die Vermutung, dass wenn der Test im zweiten Halbjahr der 7. Klasse durchgeführt wurde ein deutlich besseres Resultat erzielt werden könnte.



\section{Fragebogen}

Die verwendeten Fragen im Fragebogen aus SESSKO \citep{Schone2002} und die abgewandelten Fragen nach \citet{Dierks2014} wurden aus innere Konsistenz überprüft. Beide Skalen erreichten wie in \ref{txt:Cronbach} beschrieben eine sehr gute innere Konsistenz, insbesondere da Cronbach's $\alpha$ eher zu einer Unterschätzung der inneren Konsistenz führt \citep{Eisinga2013}. Auch durch das Weglassen einzelner Fragen würde die innere Konsistenz nicht verbessert werden (siehe Tabelle \ref{tab:SESSKO} und Tabelle \ref{tab:NatSK}). Daher kann angenommen werden, dass beide Skalen das jeweilige Selbstkonzept konsistent widerspiegeln und ausreichend Fragen zu jeder Skala vorhanden sind. 

Der Mittelwert aller Schülerinnen und Schüler beim "`Schulisches Selbstkonzept - absolut"' kann mit den Werten aus der Literatur \citep{Schone2002} verglichen werden. Dabei hat die hier untersuchte Schülergruppe ein leicht überdurchschnittliches Selbstkonzept verglichen mit der Referenzgruppe (4. - 10. Klasse in verschiedenen Deutsch Schulformen und Bundesländern.). Der Grund dafür könnte der erst kürzlich erfolgte Übertritt auf die Oberstufe und dort die Einteilung in die Sek A sein. 

\section{Unterschied zwischen den Klassen}

Vor der weiteren Analyse der Daten muss erst festgestellt werden, ob die Datensätze der einzelnen Klassen kombiniert werden dürfen. Wichtig ist dabei, dass der exakte Test nach Fischer verwendet wird und nicht der Chi-Quadrat-Test, da bei kleinen Datensätzen (wie dem hier Vorliegenden) der Chi-Quadrat-Test nicht geeignet ist \citep{Mehta1984}.

Für den exakten Fischer-Test wurden die erreichten Qualitätsstandards in den einzelnen Klassen verglichen. Die Qualitätsstandards wurden verwendet, da im Vergleich zu den Items das statistische Rauschen geringer ist und gleichzeitig nicht viel an Information verloren geht. Aus der Tabelle \ref{tab:KlassenVergleiche}, kann geschlossen werden, dass kein signifikanter Unterschied zwischen den einzelnen Klassen existiert, da alle p-Werte über 0.05 liegen.

Es dürfen daher alle Datensätze kombiniert werden, da das Erreichen eines Qualitäts-standards nicht davon abhängt in welcher Klasse ein Schüler oder eine Schülerin ist. Für alle weiteren Analysen wurden daher alle Datensätze kombiniert und nicht nach Klassen unterschieden.

\section{Ist das Abschneiden in den Tests unterschiedlich}

Nachdem gezeigt wurde, dass der ganze Datensatz insgesamt analysiert werden kann, wurde versucht die Forschungsfrage zu beantworten. Dafür ist es notwendig festzustellen, ob das Erreichen der Qualitätsstufen zwischen den unterschiedlichen Tests signifikant unterschiedlich ist.

Hierbei gibt es unterschiedliche Ergebnisse, wie in Tabelle \ref{tab:CorNiveau} ersichtlich ist. So ist die Korrelation zwischen den unbedingten Niveaus sind zwischen allen Tests signifikant. Das Spearmans $\rho$ liegt jeweils im leicht positiven Bereich, was auf eine leicht positive Korrelation hinweist. Bei dem bedingten Niveau ist nur der Test zwischen Test 201 und 305 signifikant.

Ein Grund für diese unterschiedlichen Resultate liegt vermutlich darin, das beim bedingten Niveau nur sehr wenig hohe Werte erreicht werden (siehe Tabelle \ref{tab:Niveau}). Daher kommt es zu einer geringen Datenlage bei Niveaus über 2, dies kann man auch sehr gut in den Darstellungen \ref{fig:corLev} sehen. Dies führt zu Problemen bei der Berechnung des Korrelationstestes für bedingte Niveaus, da nur sehr wenige Datenpunkte im Bereich über 2 verfügbar sind, an denen eine Verankerung stattfinden könnte. Bei besseren Schülerinnen und Schülern bei denen öfter ein höheres Niveau erreicht würde, wären diese Probleme nicht so fatal und man würde vermutlich bei beiden Niveaus eine Korrelation feststellen können.

Aufgrund der geringen Datenlage bei den bedingten Niveaus, wird der Fokus in der weiteren Arbeit auf die unbedingten Niveaus gesetzt. Aufgrund der Korrelationen zwischen diesen kann davon ausgegangen werden, dass das Erreichen eines unbedingten Niveaus in einem Test mit dem unbedingten Niveau in einem anderen Test signifikant leicht korreliert. Dies ist ein erster Hinweis darauf, dass das Erreichen eines Niveaus nicht abhängig ist in welchem Test dies erreicht wurde. Sondern rein von der Kompetenz des skalenbasierten Messens. 

\section{Rasch-Analyse}

Nach sind in der klassischen Testtheorie erste Hinweise auf die Beantwortung der Forschungsfrage gezeigt haben, wurde zusätzlich die probabilistische Testtheorie verwendet. Ein Grund diese Theorie zu verwenden ist, dass das Abgeben einer korrekten Antwort ein Zufallsprozess ist und nicht deterministisch. Aufgrund der zugrunde liegenden Daten wurde das dichotome Rasch Modell verwendet. Für das Modell wurden nur die unbedingten Qualitätsstandards verwendet, die bedingten Qualitätsstandards würden die Annahme des Rasch Modells, dass alle Items unabhängig voneinander sind, verletzt.

\subsection{Parameter-Schätzung}

Ein grosses Problem bei der Rasch-Analyse ist die Parameter-Schätzung. Das grösste Problem dabei ist, dass es im Moment in der Literatur nur zwei gängige Parameterschätzer gibt, welche im Detail analysiert wurden \citep{Fischer1995,Rost2004, Strobl2012}. Wie bereits geschrieben machen diese beiden Parameterschätzer Annahmen über die Zugrunde liegenden Daten. Bei den vorliegenden Daten kann insbesondere die Annahme über eine bestimmte Verteilung (der Einfachheit halber wird meistens eine Normalverteilung angenommen \citep{Rost2004}) der Personenfähigkeiten aufgrund der Zugrundliegeenden Daten nicht angenommen werden.

Mit beiden Parameterschätzern können zwar die Aufgaben Schwierigkeiten $\beta$ übereinstimmen geschätzt werden (siehe Darstellung \ref{fig:RaschVergleich}). Nach \citet{Rost2004} ist diese Schätzung jedoch deutlich unkritischer, wie die der Personen-Parameter. Bei den Personen-Parametern $\theta$ gibt es jedoch Unterschiede zwischen beiden Schätzen. Bei der bedingen Maximum-Likelihood-Schätzung können alle Personen-Parameter ohne Extrapolation berechnet werden. Dies ist bei der marginal Maximum-Likelihood-Schätzung nicht der Fall. Der Grund dafür liegt in der Annahme einer Normalverteilung der Personen-Parameter die der marginal Maximum-Likelihood-Schätzung zugrunde liegt. Bei grösseren Datensätzen mag diese Annahme gerechtfertigt sein, bei dem hier vorliegenden Datensatz ist dieser Schätzer jedoch nicht geeignet. Es wäre zwar prinzipiell möglich eine andere Verteilung als die Normalverteilung für die Personen-Parameter zu verwenden. Dafür müsste aber eine eigene Implementierung des Rasch-Modells vorgenommen werden, was den Rahmen dieser Arbeit sprengen würde.

Aufgrund diesem Vergleich der Parameter-Schätzungen wurde für alle weiteren Rasch Modelle der bedingen Maximum-Likelihood-Schätzer verwendet. Dessen Annahmen, dass jeder Schüler oder Schülerin mindestens ein Item richtig oder falsch beantwortet haben müssen, war jedoch bei der Aufteilung in kleinere Rasch-Modelle ein Problem. Daher sollten insbesondere für kleine Datensätze bessere Schätzer entwickelt werden, welche weniger Annahmen über die Zugrundliegeenden Daten machen. Eine Möglichkeit wäre ein Bootstrapping Algorithmus, welcher die Verteilung der Personen-Parameter aus den vorliegenden Daten selbst abschätzt und die Verteilung dann in den marginal Maximum-Likelihood Schätzer einsetzt.

\subsection{Modellkontrolle}

Nachdem der beste Parameter-Schätzer identifiziert wurde, musste das Rasch Modell jedoch noch verifiziert werden. Dafür wurde das Rasch-Modell basierend auf dem Mittelwert der Personen-Randsummen gesplittet. Aufgrund der Annahmen für das Rasch-Modell sollten dann keine signifikanten Unterschiede zwischen den beiden neuen Modellen existieren. Dies wurde vom Andersens Likelihood-Quotienten Test bestätigt, nach dem die Qualitätsstufen 4 und 5 entfernt wurden. Das Problem mit diesen beiden Qualitätsstufen ist, dass für die untersuchte Personen-Gruppe diese Standards sehr schwierig waren und sie daher kaum beantwortet wurden (siehe Tabelle \ref{tab:QS}). Aufgrund der Test Ergebnisse kann das Ausschliessen dieser Qualitätsstufen bestätigt werden, da dann ein valides Rasch-Modell vorliegt.

Zusätzlich wurden alle Qualitätsstandards noch überprüft, sowohl grafisch (siehe Darstellung \ref{fig:RaschKontrolle}), als auch mit dem Wald-Test (siehe Tabelle \ref{tab:WaldTest}). Es gab dabei kein Qualitätsstandard, welcher als ungeeignet aus dem Modell ausgeschlossen werden müsste, da er sich signifikant in den beiden Modellen unterschiedet.

Diese Resultate zeigen, dass das verwendete Rasch-Modell mit den Qualitätsstandards 1-3 valide ist. Dieses Resultat ist wichtig, da ansonsten die mit diesem Modell gewonnenen Parameter auf einer falschen Modell-Annahme beruhen würden.

\subsection{Unterschied in den Schwierigkeiten der Qualitätsstandards}

Die Schwierigkeit eines Qualitätsstandards sollte nicht davon abhängig sein, in welchem Test dieser Qualitätsstandard erreicht wurde. Dies wurde versucht mit Hilfe des Rasch-Modells zu verifizieren. Dazu wurden die \textit{item characteristic curves} (ICC) gezeichnet, siehe Darstellung \ref{fig:corLevRasch}. Diese Darstellung lassen eine qualitative Überprüfung der Schwierigkeiten dar. Man sieht das bei Qualitätsstandard 1 und 3 die beiden Test 201 und 301 sehr ähnlich sind. Bei Test 305 sind die Qualitätsstandards meistens deutlich leichter in der Schwierigkeit. Dies liegt höchstwahrscheinlich daran, dass dieser Test im Vergleich zu den anderen beiden Test leichter ist \citep{Sichau2015}. Dies sieht man auch in der Darstellung \ref{fig:PersonItemMapQ}.

Zusätzlich zu der qualitativen Überprüfung wurde noch ein Kolmogorow-Smirnow-Test durchgeführt, um festzustellen ob die Unterschiede in den Aufgaben-Parametern (siehe Tabelle \ref{tab:betaQ}) signifikant sind. Die Testergebnisse in Tabelle \ref{tab:corTestQ} zeigen, dass es keine signifikanten Unterschiede zwischen diesen Werten gibt. Wichtig ist dabei jedoch, dass diese Tests eine sehr geringe Power haben, da der Datensatz nur die Grösse von 3 hatte. Diese geringe Power zeigt sich auch in der Darstellung \ref{fig:corTestQ}.


Durch die Kombination der qualitativen und quantitativen Resultaten kann jedoch die Aussage gestützt werden, dass es keine signifikanten Unterschiede in der Schwierigkeiten der Qualitätsstandards gibt. Dies ist ein weiterer Hinweis darauf, dass das Erreichen der Qualitätsstandards 1-3 nicht davon abhängig ist, welcher Test durchgeführt wurde. 

\subsection{Unterschied in den latenten Personen-Fähigkeiten}

Nachdem es klar ist, dass die Aufgaben-Parameter sehr ähnlich sind wurden die Personen-Parameter analysiert. Hierfür wurde das Rasch-Modell aufteilt und für jeden Test ein eigenes Rasch-Modell erstellt. Hierbei gibt es nun massive Probleme mit der Parameter-Schätzung, da nun die Wahrscheinlichkeit, dass ein Schüler keinen der drei Qualitätsstandards oder alle erreicht hat, signifikant höher ist. Daher konnten viele Personen-Parameter nicht geschätzt werden.

Diese Probleme mit der Parameter-Schätzung führten auch dazu, dass das Modell nicht validiert werden konnten. Die gewonnenen Personen-Parameter basieren daher auf einem nicht validierten Modell und müssten daher mit Vorsicht interpretiert werden. Diesmal wurde daher untersucht, ob sich die Personen-Fähigkeiten zwischen den drei Rasch Modellen unterscheiden. In Tabelle \ref{tab:GOFP} und Darstellung \ref{fig:GOFP} sind die Resultate dieses Testes dargestellt. Es kann daher davon ausgegangen werden, dass die Personen-Fähigkeiten zwischen den drei Tests nicht signifikant korrelieren.

Diese Resultate sind ein Gegenindiz zu den bisher vorliegenden Resultaten, da die Personen-Fähigkeit nicht von den durchgeführten Tests abhängen sollten. Aufgrund der Datengrundlage und dem darauf basierenden Rasch-Modell sollten diese Ergebnisse jedoch nicht überbewertet werden, insbesondere da das Rasch-Modell nicht validiert werden konnte. Auch sieht man in Tabelle \ref{tab:GOFP}, dass meistens nur ein kleiner Teil der Personen-Parameter verglichen worden wurde, da der Schätzer nur für einen kleinen Teil der Personen fähig war den Personen-Parameter $\theta$ zu berechnen. Diese Ergebnisse beruhen daher Grossteils auf Problemen mit dem Parameter-Schätzer. Auch der marginale Maximum-Likelihood Schätzer hatte massive Probleme mit dem Datensatz und war noch schlechter, daher wurden dessen Ergebnisse nicht präsentiert.

Aufgrund dieser Probleme sollten diese Gegenindizien nicht überinterpretiert werden, da sie auf einer sehr schlechten Datengrundlage basieren. Dies zeigt jedoch, dass bessere Parameter-Schätzer notwendig sind, welche auch mit solchen Datensätzen umgehen können.

\subsection{Zusammenhang Rasch Modell und Fragebogen}

Das erste Rasch-Modell, bei dem alle drei Test kombiniert wurden, wurde verwendet um die latente Personen-Fähigkeit mit Resultaten des Fragebogens zu vergleichen. In Tabelle \ref{tab:CorPersonRasch} sind die Ergebnisse der Korrelations-Test dargestellt. Es gibt nur einen signifikanten Zusammenhand zwischen dem Schulversuch Selbstkonzept. 

Dieses Ergebnis ist nicht überraschend, da in der Notengebung experimentellen hand-ons Test eher eine untergeordnete Rolle spielen. Auch das SESSKO Selbstkonzept \citep{Schone2002} ist vermutlich zu generell und korreliert daher nicht mit den Personen-Fähigkeiten des Rasch-Modells. Das letzte Selbstkonzept hingegen zielt sehr genau auf das Selbstkonzept bei Schulversuchen ab, welche sehr identisch zu experimentellen hand-ons Test sind. Daher ist diese Korrelation zu erwarten. Um diese Skala jedoch zu verbessern, müsste diese noch im grösseren Rahmen validiert werden. Vor allem ist im Moment noch keine Normalverteilung der Daten gewährleistet.

\section{Videoanalyse}

In einem letzten Schritt wurden noch die Videos analysiert. Die dabei entwickelten Merkmale wurden mit den Qualitätsstandards korreliert. Wie in Tabelle \ref{tab:CorVideoQ} und Darstellung \ref{fig:corVideoQ} ersichtlich gibt es keinen Zusammenhang zwischen den im Video kodierten Merkmalen und den Qualitätsstandards, auf denen die Merkmale beruhen. Diese Ergebnisse sind zuerst enttäuschend, da die Merkmale eigentlich die Qualitätsstandards widerspiegeln sollten. Mit Beobachtungen, welche jedoch während der Test Durchführung gemacht wurden lassen sich diese Ergebnisse jedoch erklären. Viele Schülerinnen und Schüler waren während der Test Durchführung sehr auf die experimentelle Seite fokussiert und haben insgesamt sehr wenig auf den Datenbögen ausgefüllt. Dies zeigt sich auch im insgesamt eher schlechtem Abschneiden der Schülerinnen und Schüler (siehe Tabelle \ref{tab:Niveau}). Daher widerspiegeln die Qualitätsstandards nur den Teil des Experimentes wieder, welche die Schülerinnen und Schüler dokumentiert haben. 

Diese Resultate zeigen jedoch klar, dass die Kompetenz des skalenbaiserenden Messens auch der Aspekt der Dokumentation eine entscheidende Rolle spielt. Dies widerspricht sich jedoch nicht, da zu einer experimentellen Kompetenz die Fähigkeit zu Dokumentieren sehr wichtig ist. Für Schülerinnen und Schüler jedoch, welche sprachliche Schwächen haben, könnte der Einsatz von Videoanalysen hilfreich sein. Auch bei niedrigeren Schulstufen, wäre der Einsatz von Videoanalysen angebracht. Ein Nachteil ist jedoch der hohe Aufwand, welcher für die Kodierung der Videos anfällt. 

Ein weiteres Problem ist die Interpretierbarkeit der Daten. So ist es sehr schwierig aus der Darstellung \ref{fig:Messungen} gute Schlüsse zu ziehen. Diese Daten sind nur qualitativ analysierbar. Solange aber dieser Datensatz nicht grösser ist, sollten aus diesen Daten auch keine qualitativen Schlüsse gezogen werden.

\section{Zusammenfassung}

Abschliessend lässt sich sagen, dass sowohl mit der klassischen als auch mit der probabilistische Testtheorie die Forschungsfrage beantwortbar ist. Mit beiden Theorien konnten starke Hinweise darauf gefunden werden, dass bei dem vorliegenden Datensatz die Kompetenz des skalenbasierenden Messens unabhängig des fachlichen oder inhaltlichen Kontextes ist. Es gibt zwar auch Gegenanzeigen gegen dieses Resultat, bei diesen ist aber oft die Datengrundlage sehr schlecht, im Vergleich zu den unterstützenden Hinweisen. Daher kann die Forschungsfrage mit der durchgeführten Methode beantwortet werden. Bevor aber generelle Schlüsse gezogen werden sollten, müsste die Untersuchungsgruppe massiv vergrössert werden. 

Das Resultat dieser Arbeit ist daher, dass:
\begin{quote}
 Die Kompetenz des skalenbasierten Messens ist, in der vorliegenden Untersuchungsgruppe, unabhängig des fachlichen oder inhaltlichen Kontextes.
\end{quote}



\chapter{Ausblick}
\section{Datengrundlage}

Mit dieser Arbeit wurde ein erster Versuch durchgeführt, zu zeigen, dass bestimmte Kompetenzen Kontext unabhängig sind. Für einen Generalisierbarkeit der Resultate sind jedoch grössere Untersuchungsgruppen notwendig. Daher sollte ein erster Schritt dahin gehen, die Datengrundlage dieser Arbeit zu vergrössern. Damit sollten die bisher vorliegenden Hinweise stärker Hervortreten und die Korrelationen besser abschätzbar sein.

Hierbei denke ich jedoch, dass die bisherigen Ergebnisse bestärkt werden und keine gegensätzlichen Resultate gefunden werden. Die bisherigen Ergebnisse sind jedoch aufgrund der zu geringen Stichprobe nicht generalisierbar. 




\section{Videoanalyse}

In dieser Arbeit wurde versucht zusätzliche Informationen mit Videoanalyse zu generieren. Dies ist nur sehr beschränkt gelungen, insbesondere da der Hauptfokus auf quantitativen Forschungsmethoden gelegt wurde und nicht auf qualitative. Dennoch hat es sich gezeigt, dass die Ergebnisse der Videoanalyse nicht mit den über  Pen und Paper Tests erhobenen Daten übereinstimmen. Für genauere Analyse dieser Ergebnisse sollte der Fokus gezielt auf den Vergleich zwischen Videoanalyse und Pen und Paper Tests gehoben werden. 

Interessant könnten die Videoanalysen insbesondere bei sprachlich schwächeren Schülern und Schülerinnen sein, welche aufgrund sprachlicher Schwierigkeiten in Pen und Paper Tests nur schwache Leistungen zeigen. Insbesondere hier könnten Videoanalysen helfen, festzustellen, ob es wirklich ein sprachliches Problem ist oder ob diese Schüler und Schülerinnen in diesen Tests auch tatsächlich schlechtere Leistungen erbringen.

\section{Methodiken}

Auch methodisch wirft diese Arbeit weitere Fragen auf. Wie gezeigt ist insbesondere die Parameter-Schätzung des Rasch-Modells bei kleinen Datensätzen problematisch. Da die bisherigen Ansätze meistens Annahmen treffen, welche von kleinen Datensätzen nicht erfüllt werden können, treffen. Daher bräuchte es dringend neue Ansätze für die Schätzung der Personen-Parameter. Eine Möglichkeit wäre der marginale Maximum-Likelihood Schätzer. Dieser erfordert eine Annahme über die Verteilung der zugrunde liegenden Personen-Parametern. In den meisten existierenden Software wird eine Normalverteilung angenommen \citep{Rost2004, Rizopoulos2006}. Diese Annahme einer Normalverteilung ist nicht festgelegt für den marginale Maximum-Likelihood Schätzer. Dieses Problem könnte vielleicht mit einen neuen Bootstrapping-Algorithmus, welcher mit den vorliegenden Daten eine Abschätzung über die Verteilung der Personen-Paramtern macht, gelöst werden. Diese Abschätzung könnte dann als Initialisierung für den marginal Maximum-Likelihood Schätzer verwendet werden. Durch weitere Iterationen könnte dann das Ergebnis eventuell noch verbessert werden. Dies würde insbesondere bei kleinen Datensätzen das Rasch-Modell verbessern und nützlicher machen.

Ein weiteres grosse Problem der sozialwissenschaftlichen Forschung, ist die geringe Auseinandersetzung mit den verwendeten Methodiken. Auch das Verwenden von closed-source Programmen ist sehr fragwürdig, da oft die Dokumentation nicht ausreichend ist um die Ergebnisse nachvollziehen zu können (z.B. SPSS).  Meiner Meinung hat hier die Literatur in der sozialwissenschaftlichen Forschung grossen Nachholbedarf. So sollten die Verwendeten Source-Codes der Analysen frei verfügbar gemacht werden (bei der Publikation), damit andere Personen die Resultate nachvollziehen können, wie dies z.B. in der Bio-Informatik Standard ist. Diese Forderung wurde versucht in dieser Arbeit umzusetzen. Hier daher nochmals der Link zu dem Vollständigen Source-Code der Auswertung.
\github{http://git.io/buGR}




\appendix 



\backmatter	

\printbibliography[heading=bibintoc]

\chapter{Anhang}


\section{Urheberschaftsbestätigung}

Hiermit erkläre ich, dass die vorliegende Arbeit von mir eigenständig verfasst wurde und keine anderen als die von mir angegebenen Hilfsmittel verwendet wurden. Alle Stellen der Arbeit, die anderen Werken dem Wortlaut oder dem Sinn nach entnommen wurden, sind mit Angaben der Quellen als solche gekennzeichnet.

Ich nehme zur Kenntnis, dass Arbeiten, die unter Beizug unerlaubter Hilfsmittel verfasst wurden und die fremde Textteile ohne entsprechenden Herkunftsnachweis enthalten, verfolgt und geahndet werden.
\\
\\Zürich den 05. Februar 2015


\includegraphics[width=0.2\linewidth]{./graphics/UnterschriftDavid.pdf}\\
David Sichau



\section{Daten und Auswertungen}

Für die Auswertungen wurde offene Programmiersprache R\footnote{\url{http://www.r-project.org/}} verwendet. Diese ist für alle  Systeme kostenfrei verfügbar. Aller Code und die Daten dieser Masterarbeit befinden sich auf GitHub und sind frei verfügbar.
\github{http://git.io/buGR}



\section{Fragebogen}
Hier folgen die Fragebögen, welche die Schülerinnen und Schüler ausgefüllt haben. Da sie sich leicht unterschieden, wurden beide Fragebögen angehängt.

\includepdf[pages=1,pagecommand={\pagestyle{fancy}\subsection{Fragebogen am Anfang}},scale=0.8]{./graphics/Fragebogen_Anfang.pdf} 
\includepdf[pages=2-,pagecommand={\pagestyle{fancy}},scale=0.8]{./graphics/Fragebogen_Anfang.pdf} 
\includepdf[pages=1,pagecommand={\pagestyle{fancy}\subsection{Fragebogen am Ende}},scale=0.8]{./graphics/Fragebogen_Ende.pdf} 
\includepdf[pages=2-,pagecommand={\pagestyle{fancy}},scale=0.8]{./graphics/Fragebogen_Ende.pdf} 



\section{Aufgabenstellung und Kodierungen}
\label{sec:Kodierung}
Im folgenden Abschnitt befinden sich die Aufgabenstellungen der drei Tests und die Kodiermanuals.

\includepdf[pages=1,pagecommand={
\pagestyle{fancy}
\subsection{Test 201: Aufgabenstellung}
},scale=0.8]{./graphics/Test_C.pdf} 
\includepdf[pages=2-,pagecommand={\pagestyle{fancy}},scale=0.8]{./graphics/Test_C.pdf} 

\includepdf[pages=1,pagecommand={\pagestyle{fancy}\subsection{Test 201: Kodierung}},scale=0.8]{./graphics/EKN_12_M2_Kodierschema_neu.pdf} 
\includepdf[pages=2-,pagecommand={\pagestyle{fancy}},scale=0.8]{./graphics/EKN_12_M2_Kodierschema_neu.pdf} 


\includepdf[pages=1,pagecommand={\pagestyle{fancy}\subsection{Test 301: Aufgabenstellung}},scale=0.8]{./graphics/Test_B.pdf} 
\includepdf[pages=2-,pagecommand={\pagestyle{fancy}},scale=0.8]{./graphics/Test_B.pdf} 

\includepdf[pages=1,pagecommand={\pagestyle{fancy}\subsection{Test 301: Kodierung}},scale=0.8]{./graphics/EKN_12_M3_Kodierschema_neu.pdf} 
\includepdf[pages=2-,pagecommand={\pagestyle{fancy}},scale=0.8]{./graphics/EKN_12_M3_Kodierschema_neu.pdf} 


\includepdf[pages=1,pagecommand={\pagestyle{fancy}\subsection{Test 305: Aufgabenstellung}},scale=0.8]{./graphics/Test_A.pdf} 
\includepdf[pages=2-,pagecommand={\pagestyle{fancy}},scale=0.8]{./graphics/Test_A.pdf} 

\includepdf[pages=1,pagecommand={\pagestyle{fancy}\subsection{Test 305: Kodierung}},scale=0.8]{./graphics/EKN_12_M2_Kodierschema_Physik_Temperatur.pdf} 
\includepdf[pages=2-,pagecommand={\pagestyle{fancy}},scale=0.8]{./graphics/EKN_12_M2_Kodierschema_Physik_Temperatur.pdf} 

\includepdf[pages=1,pagecommand={\pagestyle{fancy}\section{Einverständnis Erklärung für Video Aufnahme}},scale=0.78]{./graphics/Erlaubnis_Videoaufnahmen.pdf} 


\end{document}
