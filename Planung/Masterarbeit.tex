
\documentclass[paper=a4, fontsize=12pt, parskip=half]{scrartcl} % A4 paper and 11pt font size


\usepackage[ngerman]{babel} % German language/hyphenation

\usepackage[utf8]{inputenc} 
\usepackage{setspace}
\onehalfspacing 
%\usepackage{cite}

\usepackage[
	natbib,
	bibencoding=utf8,
	isbn=true,
	backend=biber,
	doi=true,
	url=false,
	authordate
]{biblatex-chicago}  
\DefineBibliographyStrings{german}{%
  andothers = {et al.},
}
\addbibresource{../../../../library.bib}

\usepackage{graphicx}
\usepackage{float}
\usepackage{hyperref}

\usepackage{fancyhdr} % Custom headers and footers
%\pagestyle{fancyplain} % Makes all pages in the document conform to the custom headers and footers
\pagestyle{fancyplain}
\fancyhf{}
\renewcommand{\sectionmark}[1]{\markright{#1}}


\chead{ \fancyplain{}{\rightmark} }

\cfoot{ \fancyplain{}{\thepage} }



\usepackage{csquotes}




\setlength\parindent{0pt} % Removes all indentation from paragraphs - comment this line for an assignment with lots of text

%----------------------------------------------------------------------------------------
%	TITLE SECTION
%----------------------------------------------------------------------------------------

\newcommand{\horrule}[1]{\rule{\linewidth}{#1}} % Create horizontal rule command with 1 argument of height

\title{	
\normalfont \normalsize 
\textsc{PHZH} \\ [0.4cm] % Your university, school and/or department name(s)
\horrule{0.5pt} \\[0.3cm] % Thin top horizontal rule
\huge Konzept Masterarbeit \\ % The assignment title
\horrule{2pt} \\[0.4cm] % Thick bottom horizontal rule
}

\author{David Sichau} % Your name

\date{\normalsize\today} % Today's date or a custom date



\begin{document}

\maketitle % Print the title

\tableofcontents


\section{Konzept}

Im Rahmen des Projektes ExKoNawi der PH Zürich \citep{Gut2013a} soll untersucht werden, inwiefern die strategischen Komponenten der Messkompetenz von Lernenden vom Kontext abhängig ist. Dies erfordert eine theoretische Einarbeitung und den Begriff der Kompetenz und des Transfers und welche Verknüpfung zwischen diesen existiert. 

Der Kompetenzbegriff, welcher auf der Arbeit von \citet{Klieme2004, Weinert2001b} basiert, stellt auch die Grundlage des Kompetenzbegriffes in den internationalen Studien dar PISA \citep{PISA-KonsortiumDeuschland2004}, TIMSS \citep{Martin2003} und IGLU \citep{Bos2003}.

Interessant ist, dass trotz der Einschränkung des Kompetenzbegriffes auf spezifische Kontexte, immer noch davon ausgegangen wird, dass die Kompetenz generalisierbar ist und teilweise auf andere Situationen übertragen werden kann \citet{Hartig2006}.

\citet{Lersch2007} gibt Vorschläge wie kompetenzfördernder Unterricht gestaltet sein sollte. So fordert er, dass der Unterricht "`viel stärker von den erforderlichen Lernprozessen und -gelegenheiten her konzipiert werden müsste und eben nicht nur von einer kontinuierlichen Abfolge von Inhalten"'. Dies deckt sich mit der Forderung von \citet{Mietzel2007} für das Entkontextualiseren von Unterricht um Transferleistung zu fördern. Auch die Problemorientierung von Lerngelegenheiten wird von \citet{Lersch2007} für kompetenzfördernden Unterricht als wichtig gehalten, insbesondere fordert er, dass "`systematische Wissensvermittlung […] um variable Anwendungssituationen"' ergänzt werden sollten. Zusätzlich fordert er, dass realistische Lernsituationen angeboten werden sollten, in anderen Worten: die Lerngelegenheiten sollten mit dem Ziel der Kompetenz übereinstimmen, da der Erwerb der Kompetenz ja kontextspezifische erfolgt \citep{Klieme2004}.


\section{Fragestellungen}

Aufgrund dieser Forderungen schliesst sich kontextspezifische Kompetenzen und Trans-
ferleistungen grundsätzlich nicht gegenseitig aus. Guter komptenzorientierter Unterricht
unterstützt hingegen sogar die Fähigkeiten das Wissen zu transferieren. Diese Erkenntnis
führt nun jedoch zu der Frage:


\begin{quote}
Ist eine strategische Kompetenz von Lernenden in unterschiedlichen Kontexten
gleich verfügbar?
\end{quote}
Diese Frage verknüpft sehr stark den Begriff des Transfers mit dem Kompetenzbegriff.
Unter einer strategischen Kompetenz versteht man das Anwenden einer Lösungsstrategie
auf verschiedene fachliche Inhalte und sollte unabhängig vom fachlichen Kontext sein. Dies
bedeutet im Bezug auf den Transferbegriff jedoch, dass die Lernenden eine Transferleistung
erbringen müssen, da sie diese Kompetenz auf verschiedene Kontexte anwenden müssen.
In der vorliegenden Arbeit soll nun untersucht werden, inwiefern die strategische Kompetenz des Messens sich in unterschiedlichen Kontexten unterscheidet.


Es gibt jedoch noch weitere Forschungsfragen, die sich aus dieser Fragestellung ergeben. So soll auch untersucht werden, ob es Kriterien gibt, welche eventuelle Unterschiede in der Verfügbarkeit des Wissens einordnen könnten.

Die Daten sollen zuerst einer explorativen Datenanalyse unterzogen werden, um noch unbekannte oder nicht erwartende Zusammenhänge aufzuzeigen. Darauf aufbauend werden die Daten analysiert, um das theoretische Rahmenkonstrukt zu überprüfen.

\section{Untersuchungsdesign}

Aufbauend auf diesem Rahmenkonstrukt und der Fragestellung wird eine Untersuchung an 4 Klassen durchgeführt. Diese Klassen werden jeweils drei hand-on Experimentiertest zum Problemtyp skalenbasiertes Messen \citep{Metzger2013} erhalten, wobei jedoch der Kontext, indem sie die Kompetenz des Messens anwenden unterschiedlich ist. Die Schülerinnen und Schüler jeder Klasse werden in einer Doppelstunde jeweils drei hand-on Experimentiertest durchführen. Zusätzlich werden sie einen standardisierten Fragebogen ausfüllen zum schulischen und zum Selbstkonzept zu hand-on Experimenten. 

Zusätzlich zu den Test werden Videoanalysen einzelner SuS durchgeführt und standardisierte Fragen zu dem schulischen und naturwissenschaftlichem Selbstkonzept erhoben. Es sollen insgesamt aus jeder Klasse je 2 SuS per Video aufgenommen werden und die Daten dazu erhoben werden.

\section{Bisher untersuchte Literatur}

Bisher wurde folgende Literatur untersucht. Die Literatur Recherche ist noch nicht abgeschlossen und daher wird im Rahmen der Arbeit noch zahlreiche weitere Literatur hinzukommen.



\printbibliography[heading=none]





\end{document}
