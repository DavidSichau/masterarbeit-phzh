

In dieser Forschungsarbeit wurde die Validierung eines Testes durchgeführt. Aufgrund der zu geringen Datengrundlage von 72 Schülerinnen und Schüler konnte diese Validierung nicht zufriedenstellend durchgeführt werden. Vor einem grösseren Einsatz der Tests (nicht nur des neu entwickelten) sollte daher die Datengrundlage vergrössert werden. Insbesondere sollten auch Schülerinnen und Schüler aus höheren Schulstufen untersucht werden.

Es solle auch noch weiter untersucht werden, auf welcher Ebene die Validierung durchgeführt werden sollte. Die tiefste Ebene der einzelnen Items weisst ein schlechte Signal/Noise Verhältnis auf. Daher sollte wie in \citet{Sichau2015a} die Validierung auch auf der Ebene der Qualitätsstandards oder der Qualitätsniveaus durchgeführt werden, da dort das Signal/Noise Verhältnis besser ist.

Es gibt jedoch erste starke Indizien dafür, dass der neu entwickelte Test eine geringere Schwierigkeitsstufe aufweist, als die beiden existierenden Test. Zumindest für die Untersuchungsgruppe. Ein Grund dafür könnte im höheren Alltagsbezug des neu entwickelten Test liegen. Im Vergleich zum Auflösen von Salz oder der Reissfestigkeit eines Fadens ist das Mischen von Wasser alltäglicher.