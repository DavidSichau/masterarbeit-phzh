Ein Problem dieser Analyse war, dass im Vergleich zur Masterarbeit die Untersuchung auf Ebene der Items durchgeführt wurde. Ein Problem dabei ist das grosse Rauschen in diesen Daten und die grosse Anzahl Items, welche nur von einer sehr geringe Anzahl Schülerinnen und Schüler beantwortet wurden, aufgrund der hohen Aufgabenschwierigkeit.

Es war dennoch möglich gewisse Unterschiede zwischen den Test aufzuzeigen.

\subsection{Unterschiede zwischen den Tests}
Mit der Korrelationsanalyse konnte gezeigt werden (siehe Tabelle \ref{tab:cor}), dass der Test 201 und 301 und zwischen den Tests 201 und 305 in 4 von 11 eine Korrelation aufweist. Zwischen Test 301 und 305 konnte nur in einem Item eine Korrelation festgestellt werden.

Dies ist ein Anzeichen dafür, dass der neu entwickelte Test ähnlicher zu Test 201 ist, als zu Test 301. Dies kann mit der Test-Konstruktion zusammenhängen, da der Test 201 die Basis für Test 305 dargestellt hat.

\subsection{Rasch Analyse}

Bei der Rasch Analyse war ein grosses Problem, dass auf der Item Ebene viele Items nur von sehr wenigen Schülerinnen und Schülern gelöst werden konnten. Dies sieht man auch in der Darstellung \ref{fig:PIM}. Die meisten Items waren für einen Grossteil der Schülerinnen und Schüler zu schwer.

Dies machte die Validierung des Rasch Modells unmöglich bei einer Signifikanz von $< 5\%$. Wenn das Signifikanzniveau auf 11\% erhöht würde, könnte das Modell validiert werden. 

Mit dem erstellten partial-credit Modell konnten einige Items gezeigt werden, welche ungeeignet für diese Schulstufe sind und daher besser aus den Tests entfernt werden sollten. Die ungeeigneten Items sind oft die Items, welche einen besonders hohen Schwierigkeitsgrad aufweisen und daher nur von einer geringen Anzahl von Schülerinnen und Schülern gelöst wurden. Es ist daher fraglich, ob diese Tests auf der Schulstufe der 7. Sek A geeignet sind und nicht besser bei höheren Schulstufen angewendet werden sollten.

In der Darstellung \ref{fig:PIM} kann man erkennen, dass einige Items des neuen Tests 305 im Schwierigkeitslevel deutlich niedriger sind wie die identischen Items in Test 201 oder 301. Es kann daher davon ausgegangen werden, dass der neue Test eine geringere Schwierigkeit aufweist, wie die alten Tests 201 und 301. Ein Grund könnte darin liegen, dass das Mischen von Wasser einen höheren Alltagsbezug für die meisten Schülerinnen und Schüler aufweist, wie das Auflösen von Salz oder die Kraft mit einer Feder zu messen. 

