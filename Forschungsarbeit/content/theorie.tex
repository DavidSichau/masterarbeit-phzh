
\subsection{Kompetenz des skalenbasierten Messens}

Die theoretische Grundlage des Testes stellt die Kompetenz des skalenbasierten Messens dar. Die Definition dieser Kompetenz basiert auf der Arbeit von \citet{Munier2013}. In dieser Kompetenz geht es darum  "`quantitative Grössen mit gegebenen Messinstrument genau [zu] messen"'\citep{Gut2013a}. Diese Kompetenz kann in drei Teilbereiche aufgegliedert werden. Zum einen ist die Wahl des Messinstrumentes, welches für eine Messung geeignet ist, eine wichtige Teilkompetenz. Nur mit einem geeigneten Messinstrument kann die Messung durchgeführt werden. Zusätzlich sind bei genauen Messungen die Wiederholung der Messung elementar. Nur mit Messwiederholungen lässt sich die Genauigkeit des Resultates abschätzen. Daher ist die Messwiederholung ein weiterer wichtiger Teilaspekt der Kompetenz des Messens. Der dritte Teilaspekt ist die korrekte Messung. Wenn das Messinstrument nicht korrekt verwendet wird, kann kein korrektes Messergebnis erhoben werden. Diese Teilkompetenzen ergeben zusammen die Kompetenz des skalenbasierten Messens \citep{Munier2013,Gut2013a}.

Die Kompetenz des skalenbasierten Messens beruht auf dem Kompetenzmodell von \citet{Gott1996}, bei welchem Kompetenzen aus transferfähiges Strategiewissen und kontextspezifisches Fachwissen bestehen \citet{Gott2002}.

\subsection{ExKoNawi}

Beim Projekt ExKoNawi der PH-Zürich \citep{Metzger2013} geht es darum hands-on Experimentiertests zu entwickeln, welche verschiedene Kompetenzen von Schülern und Schülerinnen auf der Sekundarstufe I in Schweizer Schulen zu messen. Eine der gemessenen Kompetenzen ist die Kompetenz des skalenbasierten Messens. Für diese Kompetenz gibt es bereits existierende Tests \citep{Metzger2013, Gut2013a, Hild2014a, Hild2014}. Diese Tests sind in verschiedenen Kontexten angesiedelt, messen aber alle die Kompetenz des skalenbasierten Messens. Im Test 201 wird die Temperaturveränderung von Wasser bei auflösen eines Salzes gemessen. Im Test 301 wird gemessen, bei welcher Kraft ein Faden reisst.


\section{Testentwicklung}

Wenn wir nun den fachlichen Kontext untersuchen der Test analysieren, so ist der Test 201 im Bereich Chemie angesiedelt der Test 301 jedoch in der Physik. Beim inhaltlichen Kontext geht es um eine Temperaturmessung und einmal um eine Kraftmessung. Um nun messen zu können ob es einen Unterschied bei der Kompetenz des skalenbasierten Messens zum einen bei fachlichen und zum anderen beim inhaltlichen Kontext gibt, musste ein weiterer Test entwickelt werden. Dieser Test mit der Nummer 305 musste daher fachlich in der Physik angesiedelt sein. Inhaltlich sollte dieser Test eine Temperaturmessung enthalten.

Da dieser Test auf der Sek I in der Klasse 7 durchgeführt werden sollte, musste das Themengebiet und die Aufgabe leicht verständlich sein. Zusätzlich wird der Test extern durchgeführt, daher sollte der Materialaufwand gering sein und eine schnelle Einrichtung von Arbeitsplätzen möglich sein. Daher wurde entschieden als Aufgabe die möglichst genaue Bestimmung der Mischtemperatur von kaltem und warmen Wasser zu stellen. Der gesamte Test wurde sehr stark an der Aufgabenstellung von Test 201 angelehnt. Auch das Kodierschemata wurde vom existierenden abgeleitet und angepasst. Die Tests finden sie im Anhang unter Abschnitt \ref{sec:Kodierung}.

